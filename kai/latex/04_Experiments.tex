\chapter{Experiments}
\label{Experiments}


\section*{Model select}
Approximately a decade has passed since Goodfellow introduced Generative Adversarial Networks (GANs), 
during which numerous variants of GAN models have been developed. For the purpose of this study, 
I have selected seven distinct GAN models for examination and minimal implementation: Standard GANs, 
Conditional GANs, Auxiliary Classifier GANs, Cycle GANs, Domain Transfer Network GANs, Coupled GANs, 
and Style GANs. The primary criterion for selection is training time, with only one model chosen for 
further in-depth analysis. Due to the extensive training time required, I ultimately selected Standard GANs for detailed study.


\section*{Model Structure}

In this step, I compared two types of standard GANs. The first model utilized dense layers, while the 
second model employed convolutional layers. The structure of the generator and discriminator was modified 
accordingly, while all other hyperparameters were kept constant. Both models were trained on the MNIST 
dataset for 1000 epochs. Upon comparing the generated images, it was observed that the GAN with the convolutional 
neural network (CNN) architecture outperformed the one with the dense layer architecture in terms of image quality. 
Consequently, I selected the standard GAN model implemented with convolutional layers for further experiments.

\begin{figure}[H]
    \centering
    \begin{subfigure}[b]{0.45\linewidth}
        \centering
        \includegraphics[width=\linewidth]{./Images/generator_dense.jpg}
        \caption{Generator with dense layer}
        \label{fig:Dense}
    \end{subfigure}
    \hspace{0.05\linewidth}
    \begin{subfigure}[b]{0.45\linewidth}
        \centering
        \includegraphics[width=\linewidth]{./Images/generator_cnn.jpg}
        \caption{Generator with convelution layer}
        \label{fig:Conv2D Transpose}
    \end{subfigure}
    \caption{The structure for generator}
    \label{fig:combined}
\end{figure}


\begin{figure}[H]
    \centering
    \begin{subfigure}[b]{0.45\linewidth}
        \centering
        \includegraphics[width=\linewidth]{./Images/discriminator_dense.jpg}
        \caption{Discriminator with dense layer}
        \label{fig:Dense}
    \end{subfigure}
    \hspace{0.05\linewidth}
    \begin{subfigure}[b]{0.45\linewidth}
        \centering
        \includegraphics[width=\linewidth]{./Images/discriminator_cnn.jpg}
        \caption{Discriminator with convelution layer}
        \label{fig:Conv2D Transpose}
    \end{subfigure}
    \caption{The structure for discriminator}
    \label{fig:combined}
\end{figure}


\begin{figure}[H]
    \centering
    \begin{subfigure}[b]{\linewidth}
        \centering
        \includegraphics[width=0.7\linewidth]{./Images/generate_image_by_dense_layer.jpg}
        \caption{Images generate by dense structure}
        \label{fig:Dense}
    \end{subfigure}
    \vspace{0.05\linewidth} 
    \begin{subfigure}[b]{\linewidth}
        \centering
        \includegraphics[width=0.7\linewidth]{./Images/generate_image_by_Convelution_layer.jpg}
        \caption{Images generate by convelution structure}
        \label{fig:Conv2DTranspose}
    \end{subfigure}
    \caption{Model performance 1000 epochs}
    \label{fig:combined}
\end{figure}

\section*{Data Augmentation}

I conducted three training runs of the standard GAN model, both with and without data augmentation, 
and compared the FID scores. The results indicate that data augmentation generally led to lower optimization performance.


\begin{figure}[H]
    \centering
    \begin{subfigure}[b]{\linewidth}
        \centering
        \includegraphics[width=0.8\linewidth]{./Images/standard_GAN_without_data_augementation1.jpg}
        \caption{Standard GAN without data augmentation 1}
        \label{fig:Dense}
    \end{subfigure}
    \vspace{0.05\linewidth} 
    \begin{subfigure}[b]{\linewidth}
        \centering
        \includegraphics[width=0.8\linewidth]{./Images/standard_GAN_without_data_augementation2.jpg}
        \caption{Standard GAN without data augmentation 2}
        \label{fig:Conv2DTranspose}
    \end{subfigure}
    \begin{subfigure}[b]{\linewidth}
        \centering
        \includegraphics[width=0.8\linewidth]{./Images/standard_GAN_without_data_augementation3.jpg}
        \caption{Standard GAN without data augmentation 3}
        \label{fig:Conv2DTranspose}
    \end{subfigure}
    \caption{The FID scores for standard GAN without data augmentation}
    \label{fig:combined}
\end{figure}


\begin{figure}[H]
    \centering
    \begin{subfigure}[b]{\linewidth}
        \centering
        \includegraphics[width=0.8\linewidth]{./Images/standard_GAN_with_data_augementation1.jpg}
        \caption{Standard GAN with data augmentation 1}
        \label{fig:Dense}
    \end{subfigure}
    \vspace{0.05\linewidth} 
    \begin{subfigure}[b]{\linewidth}
        \centering
        \includegraphics[width=0.8\linewidth]{./Images/standard_GAN_with_data_augementation2.jpg}
        \caption{Standard GAN with data augmentation 2}
        \label{fig:Conv2DTranspose}
    \end{subfigure}
    \begin{subfigure}[b]{\linewidth}
        \centering
        \includegraphics[width=0.8\linewidth]{./Images/standard_GAN_with_data_augementation3.jpg}
        \caption{Standard GAN with data augmentation 3}
        \label{fig:Conv2DTranspose}
    \end{subfigure}
    \caption{The FID scores for standard GAN with data augmentation}
    \label{fig:combined}
\end{figure}

\section*{Apply New Dataset}