\chapter{Theoretical Background}
\label{Theoretical Background}

\section*{Generative Adversarial Nets}

Generative Adversarial Networks (GANs) consist of a generator and a discriminator, 
both implemented using artificial neural networks. The parametrization of GANs involves 
defining the network structure of these components and initializing their weights and biases \citep{10.1007/s10928-021-09787-4}. 
The success of GANs relies on balancing the training of these two networks, where the 
generator aims to produce samples that mimic real data distributions, while the discriminator 
learns to differentiate between real and generated samples \citep{10.1109/taslp.2017.2761547}. 
The training process involves iteratively updating the weights and biases of the networks through 
adversarial training, where the generator tries to deceive the discriminator, and the discriminator 
aims to accurately classify samples \citep{10.48550/arxiv.1802.05637}.

\begin{equation}
    \min_{G} \max_{D} V(D, G) = \mathbb{E}_{x \sim p_{data}(x)} [\log D(x)] + \mathbb{E}_{z \sim p_{z}(z)} [\log(1 - D(G(z)))].
\end{equation}

The GAN network parameters play a crucial role in determining the quality and diversity of 
the generated samples \citep{10.1007/s10928-021-09787-4}. The optimization process in GANs 
typically involves minimizing a min-max function to ensure the generator produces samples that 
are indistinguishable from real data \citep{10.1109/taslp.2017.2761547}. Maintaining this balance 
during training is essential for achieving high-quality sample generation \citep{10.1007/s10928-021-09787-4}. 
The discriminator's role is to provide feedback to the generator by acting as a critic, 
guiding the generator towards producing more realistic samples \citep{10.48550/arxiv.1802.05637}.

Furthermore, the structure of GANs, with the generator and discriminator networks, is fundamental to the overall performance of the model \citep{10.1109/taslp.2017.2761547}. The generator takes a latent noise vector as input and aims to generate samples that match the distribution of real data \citep{10.48550/arxiv.1908.05861}. On the other hand, the discriminator is responsible for distinguishing between real and generated samples, providing crucial feedback for the training process \citep{10.1002/mp.14062}.

