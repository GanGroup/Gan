\chapter{Introduction}


Generative Adversarial Networks (GANs) have emerged as a transformative tool in generative modeling, framing the problem as a competition between two networks: a generator that creates synthetic data from noise, and a discriminator that differentiates between real and generated data \citep{10.48550/arxiv.1704.00028}. Since their introduction in 2014, GANs have found applications in fields such as materials science, radiology, and computer vision \citep{10.1002/mgea.30}, \citep{10.1016/j.media.2019.101552}, \citep{10.1016/j.artmed.2020.101938}. For instance, CycleGAN has been applied in medical imaging, enhancing tasks like liver lesion classification through synthetic image augmentation, outperforming traditional methods in sensitivity and specificity \citep{10.1016/j.neucom.2018.09.013}. Similarly, StyleGAN has shown effectiveness in image deformation and style transfer \citep{10.1109/iccv.2019.00453}, further expanding the reach of GANs in the computer vision field, where they generate data without explicitly modeling probability density functions \citep{10.1016/j.media.2019.101552}.

However, despite their success, training GANs presents notable challenges, including issues like mode collapse, training instability, and the high computational demands required for effective performance. Evaluating GANs is also complex, as traditional metrics such as accuracy are insufficient to measure the quality and diversity of generated data. These difficulties have driven the development of various architectures and training methods aimed at improving the stability and effectiveness of GANs.

The objective of this thesis is to contribute to a clearer understanding of the factors that influence GAN performance, particularly in generating realistic images. For this purpose, I use the Animal Faces-HQ (AFHQ) dataset, containing 16,130 high-resolution images at 512×512 pixels, to train a GAN model specifically for generating realistic cat images. The high resolution presents both opportunities and challenges, as it requires careful attention to the model’s architecture and training to avoid overfitting or underfitting.

This thesis is organized as follows: Chapter 2 provides an overview of previous models for image generation, including Deep Boltzmann Machines, Variational Autoencoders, and Noise Contrastive Estimation. Chapter 3 covers the theoretical background of GANs, focusing on their objective functions, training dynamics, and performance evaluation. Chapter 4 describes the experimental work, including model selection, architectural exploration, and the impact of data augmentation, along with the application of the GAN model to the AFHQ dataset and then presents the results, discusses their implications, and suggests potential avenues for future research.
