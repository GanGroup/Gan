\chapter{Introduction}


Generative Adversarial Networks (GANs) have emerged as a powerful class of generative models that revolutionize 
generative modeling by framing it as a game between two networks: a generator network that produces synthetic data 
from noise and a discriminator network that distinguishes between the generated data and real data \citep{10.48550/arxiv.1704.00028}. 
These networks, introduced in 2014, have found applications in various fields, including materials science, 
radiology, and computer vision \citep{10.1002/mgea.30}, \citep{10.1016/j.media.2019.101552}, \citep{10.1016/j.artmed.2020.101938}. 
For example, CycleGAN has been effectively applied in the medical field, notably in medical imaging tasks. 
It has enhanced liver lesion classification through GAN-based synthetic medical image augmentation, surpassing traditional data 
augmentation methods in sensitivity and specificity \citep{10.1016/j.neucom.2018.09.013}. 
Abdal et al. (2019) demonstrated the effectiveness of StyleGAN in tasks such as image deformation and style transfer \citep{10.1109/iccv.2019.00453}. 
GANs have gained significant attention in the computer vision community due to their ability to generate data without explicitly 
modeling the probability density function \citep{10.1016/j.media.2019.101552}.

Despite these advances, training GANs remains a challenging task due to issues such as mode collapse, instability during training, 
and the requirement for substantial computational resources. Additionally, evaluating the performance of GANs is not straightforward, 
as traditional metrics such as accuracy do not adequately capture the quality or diversity of the generated data. This complexity has 
led to the development of various GAN architectures and training techniques aimed at improving the stability and quality of generated images.

The purpose of this thesis is to contribute to a better understanding of the factors influencing the performance of GANs, 
particularly in the context of generating realistic images. To achieve this, I will utilize the Animal Faces-HQ (AFHQ) dataset, 
which comprises 16,130 high-quality images with a resolution of 512×512 pixels, to train a standard GAN model specifically designed 
for generating realistic cat pictures. The high resolution of these images presents both an opportunity and a challenge, as it requires 
careful consideration of model structure and training strategies to avoid issues such as overfitting or insufficient model capacity.

This thesis is structured as follows: Chapter 2 provides an overview of historical models of image generation, 
including Deep Boltzmann Machines, Variational Autoencoders, and Noise Contrastive Estimation. Chapter 3 discusses 
the theoretical background of GANs, focusing on their objective functions, the training process, and methods 
for evaluating their performance. Chapter 4 presents the experimental work conducted, including model selection, 
architecture exploration, and the impact of data augmentation, followed by the application of the GAN model to the 
AFHQ dataset. Finally, Chapter 5 discusses the results, highlights the implications of the findings, and suggests 
directions for future research.

