\chapter{History Models of Image Generate}
\label{Related Work}

\section*{Deep Boltzmann Machines}

Deep Boltzmann Machines (DBMs) are sophisticated generative models that excel in learning intricate data representations 
in an unsupervised manner. Unlike traditional feedforward networks, DBMs are structured with multiple layers of stochastic, 
binary latent variables, where each layer captures progressively abstract features of the input data. This architecture is 
characterized by fully connected layers, with no direct connections between units within the same layer, which enables DBMs 
to effectively model complex dependencies inherent in the data \citep{10.1007/978-3-642-40728-4_14}. The undirected nature 
of the connections in DBMs allows for the propagation of uncertainties across layers, enhancing their capability in feature 
learning and unsupervised pre-training \citep{10.1007/978-3-642-40728-4_14}\citep{10.48550/arxiv.1203.3783}.

However, the training of DBMs presents significant challenges, primarily due to their reliance on Markov Chain Monte Carlo (MCMC) methods, 
such as Gibbs sampling, for approximate inference. These methods are essential for estimating gradients and the intractable partition 
function, which is vital for updating model parameters. The computational overhead introduced by MCMC methods often results in 
slow convergence and increased complexity, particularly when applied to large-scale datasets \citep{10.48550/arxiv.2303.10728}. 
The requirement for numerous sampling steps to achieve equilibrium further complicates the training process, making DBMs less 
accessible compared to other generative models, such as Variational Autoencoders (VAEs).

To mitigate these training difficulties, DBMs often utilize layer-wise pre-training, which initializes model parameters 
before the entire network is fine-tuned. This approach has been shown to enhance the efficiency of the learning process
by providing a sensible initialization for the weights and variational inference \citep{10.1162/neco_a_00311}. Despite 
these strategies, the inherent complexity associated with MCMC methods continues to pose challenges in scaling DBMs 
effectively, particularly in comparison to more straightforward generative models like VAEs \citep{10.48550/arxiv.2303.10728}. 
Recent advancements have explored alternative training methodologies, including the use of specialized hardware systems to improve 
the efficiency of sampling tasks, thereby addressing some of the computational hurdles associated with traditional training approaches \citep{10.48550/arxiv.2303.10728}.



\section*{Variational Autoencoders}
Variational Autoencoders (VAEs) are generative models in machine learning that excel in tasks like image generation and data 
representation by learning the underlying structure of data through a latent space. 

The architecture of VAEs consists of an encoder that maps input data into a latent space, where the latent variables are 
typically assumed to follow a Gaussian distribution. This assumption simplifies the learning process, as it allows for 
the use of techniques such as the reparameterization trick, which enables backpropagation through stochastic layers \citep{10.1561/2200000056}. 
The decoder then reconstructs the input data from the latent variables, ensuring that the model captures the essential 
features of the data distribution. The loss function of a VAE combines reconstruction loss with a Kullback-Leibler (KL) 
divergence term, which regularizes the latent space and encourages the model to learn a smooth and continuous representation \citep{10.3390/jimaging4020036}.

In comparison to DBMs, VAEs offer several advantages, including simpler training dynamics and better scalability. DBMs 
often require complex sampling methods, such as Gibbs sampling, which can be computationally intensive and slow to converge. 
In contrast, VAEs can be trained using standard gradient descent methods, making them more accessible for 
practitioners \citep{10.1561/2200000056}\citep{10.1109/access.2020.2977671}. Furthermore, the structured latent space of 
VAEs not only allows for efficient sampling but also supports various applications, such as image generation, anomaly detection, 
and data imputation, where the ability to interpolate between data points is crucial \citep{10.1088/2632-2153/ab80b7}\citep{10.48550/arxiv.2002.10464}.

However, VAEs face limitations in modeling discrete data as they require back-propagation through hidden units, which poses challenges 
in handling such data types effectively \citep{10.48550/arxiv.1909.13062}. Despite this limitation, VAEs have been extensively used to 
represent high-dimensional complex data by learning a low-dimensional latent space in an unsupervised manner \citep{10.48550/arxiv.2106.06500}.


\section*{Generative Stochastic Networks}


Noise Contrastive Estimation (NCE) is a technique used in training generative models by distinguishing between data and noise samples \citep{10.21437/interspeech.2016-1295}. It has been applied in various fields such as speech recognition and language modeling due to its ability to handle large vocabularies efficiently \citep{10.1609/aaai.v32i1.11967}. NCE addresses the computational challenges posed by traditional methods like softmax by transforming the estimation problem into a binary classification task \citep{10.18653/v1/e17-2003}. By discriminating between observed data and artificially generated noise, NCE enables the estimation of non-normalized models effectively \citep{10.48550/arxiv.1805.07516}.

One of the key limitations of NCE is the need to evaluate and backpropagate two probability densities, one for the noise distribution and the other for the model distribution \citep{10.18653/v1/e17-2003}. Despite this limitation, NCE has proven to be a highly effective approach for unsupervised representation learning using deep networks \citep{10.48550/arxiv.2205.01789}. The method has also been extended to flow models for energy-based models, where the update is based on noise contrastive estimation \citep{10.1109/cvpr42600.2020.00754}.

In the context of training large vocabulary neural language models, NCE has been shown to be a sampling-based technique that offers speed improvements \citep{10.18653/v1/p16-1186}. Researchers have explored different strategies to enhance the training algorithms, including investigating noise contrastive estimation and diagonal contexts for further speed improvements \citep{10.3115/v1/n15-1083}. Additionally, the use of noise-contrastive estimation has been linked to self-supervised tasks in state-of-the-art methods \citep{10.48550/arxiv.2203.01110}.

In conclusion, Noise Contrastive Estimation (NCE) is a valuable technique for training generative models efficiently, particularly in scenarios with large vocabularies. While it has some limitations related to the evaluation of multiple probability densities, researchers continue to explore and extend NCE to address various challenges in unsupervised representation learning.


\section*{Predictability minimization}

Machine learning method for GAN



Predictability minimization is a technique that involves training two neural networks competitively to encourage the hidden units of the neural network to be independent of each other. Unlike Generative Adversarial Networks (GANs), where the competition is the sole training criterion, predictability minimization uses a regularizer to promote independence among the hidden units \citep{10.1145/2640087.2644155}.

In the context of predictive coding, predictability minimization aims to reduce prediction errors by minimizing the mismatch between incoming sensations and predictions established through experience. This process involves deviance processing, which is part of an inference process where prediction errors are minimized at different levels of the auditory hierarchy \citep{10.3389/fnhum.2015.00505}.

Regularization techniques play a crucial role in predictability minimization by penalizing model complexity to prevent overfitting. These methods typically involve balancing prediction errors on training data against regularization to optimize the model's performance \citep{10.1186/1471-2105-16-s16-s2}. Regularization functions often impose constraints on the model to ensure smooth transitions and prevent over-parameterization, thus aiding in minimizing prediction errors \citep{10.1007/978-1-4615-5339-7_102}.

Overall, predictability minimization, through the use of regularization techniques and competitive training of neural networks, aims to enhance the independence of hidden units and reduce prediction errors by optimizing model complexity and promoting smoother transitions within the model.


