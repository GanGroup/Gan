\documentclass[12pt,a4paper]{report}

\setcounter{secnumdepth}{3} %The level of sections that get numbered
\setcounter{tocdepth}{3} %The level of sections to appear in ToC
\usepackage{url}
\usepackage{lscape}
\usepackage{graphicx}
\usepackage{pdfpages}
\usepackage{listings}
\usepackage{setspace}
\usepackage[numbers]{natbib}
\usepackage{amsfonts} % This package is needed for \mathbb
\usepackage{float}
\usepackage[hidelinks]{hyperref}
\usepackage{amsmath}
\usepackage{subcaption}
\usepackage{multirow}
\usepackage{colortbl}
\usepackage{xcolor}
\usepackage{diagbox}
\usepackage{listings} % For code listings


\lstdefinestyle{mypython}{
    language=Python,
    basicstyle=\ttfamily\footnotesize,
    keywordstyle=\color{blue},
    commentstyle=\color{green},
    stringstyle=\color{red},
    numbers=left,
    numberstyle=\tiny\color{gray},
    frame=single,
    breaklines=true,
    backgroundcolor=\color{lightgray!20}
}



%Date formatting Package
\usepackage[ddmmyyyy]{datetime}

\usepackage{wrapfig}


\begin{document}
\begin{titlepage} \vspace{1cm}

\begin{center}
\textbf{\Huge Generating images using a Deep Convolutional GANs}
\end{center}

\vspace{1.0cm}

\begin{figure}[h]
    \centering
    \includegraphics[scale = 0.7]{Images/UCC_logo} 
\end{figure}

\vspace{1.0cm}

\begin{center}
\textbf{\Large Author:} \Large{Kai Deng} {\LARGE\par}
\textbf{\Large Supervisor:} \Large{Serhiy Yanchuk} {\LARGE\par}
\end{center}
\vspace{1.0cm}

\begin{center}
\Large
A thesis submitted in partial fulfilment of the requirements for the degree of\\ MSc Mathematical Modelling and Machine Learning
\vfill{}
\Large{School of Mathematical Sciences, \\
University College Cork, \\
Ireland \\
\vspace{0.4cm}
September 2024}
\end{center}

\end{titlepage}

\onehalfspacing

\chapter*{Declaration of Authorship}

This report is wholly the work of the author, except where explicitly
stated otherwise. The source of any material which was not created
by the author has been clearly cited. \\
\medskip{}

\textbf{Date:} \quad \today

\medskip{}

\textbf{Signature: Kai Deng} 

\chapter*{Acknowledgements}

I would like to express my deepest gratitude to my supervisor, Dr Serhiy Yanchuk, 
for his invaluable guidance, continuous support, and insightful feedback throughout the 
course of my research. His expertise and encouragement have been instrumental in the 
successful completion of this thesis.

\begin{abstract}
    Generative Adversarial Networks (GANs) have become a pivotal tool in the field of generative modeling, 
    offering a novel approach to generating realistic data through the adversarial training of a generator 
    and a discriminator. This thesis investigates the performance of GANs in generating high-quality images, 
    with a particular focus on the challenges associated with model structure, training stability, and evaluation metrics.
    
    In this study, I employed the Animal Faces-HQ (AFHQ) dataset, consisting of 16,130 high-resolution images, 
    to train a standard GAN model aimed at generating realistic cat images. The research explored the impact of
     various architectural choices, including the use of convolutional layers versus dense layers, as well as 
     the effects of data augmentation techniques on model performance. The experiments revealed that convolutional 
     architectures significantly outperformed dense ones, highlighting the importance of spatial feature extraction
     in image generation tasks.
    
    The study also examined the role of data augmentation in GAN training, finding that while augmentation can 
    introduce useful variability into the training data, it may also lead to optimization challenges if not carefully 
    applied. The final model, trained on downscaled images due to hardware limitations, was able to produce cat images, 
    demonstrating the effectiveness of GANs in image synthesis.
    
    The findings contribute to a deeper understanding of how GAN architecture and training strategies influence 
    the quality of generated images, offering insights for future research and practical applications in fields 
    requiring high-quality image generation. The study concludes by suggesting directions for further exploration, 
    including the potential benefits of more advanced GAN variants and higher-resolution training data.
\end{abstract}

\tableofcontents{}

\listoffigures{}

\newpage
\chapter{Introduction}

Generative Adversarial Networks (GANs) have emerged as a powerful class of generative models that revolutionize 
generative modeling by framing it as a game between two networks: a generator network that produces synthetic data 
from noise and a discriminator network that distinguishes between the generated data and real data \citep{10.48550/arxiv.1704.00028}. 
These networks, introduced in 2014, have found applications in various fields, including materials science, 
radiology, and computer vision \citep{10.1002/mgea.30}, \citep{10.1016/j.media.2019.101552}, \citep{10.1016/j.artmed.2020.101938}. 
For example, CycleGAN has been effectively applied in the medical field, notably in medical imaging tasks. 
It has enhanced liver lesion classification through GAN-based synthetic medical image augmentation, surpassing traditional data 
augmentation methods in sensitivity and specificity \citep{10.1016/j.neucom.2018.09.013}. 
Abdal et al. (2019) demonstrated the effectiveness of StyleGAN in tasks such as image deformation, style transfer \citep{10.1109/iccv.2019.00453}. 
GANs have gained significant attention in the computer vision community due to their ability to generate data without explicitly 
modeling the probability density function \citep{10.1016/j.media.2019.101552}.

The purpose of this paper is to gain a comprehensive understanding of Generative Adversarial Networks (GANs).
To achieve this, I will utilize the Animal Faces-HQ dataset, which comprises 16,130 high-quality images with a 
resolution of 512×512 pixels, to train a standard GAN model specifically designed for generating realistic cat pictures.

The Structure of this thesis.



\newpage
\chapter{Historical Models of Image Generation}

In this chapter, I aim to provide an overview of three important generative modeling techniques: Noise Contrastive Estimation (NCE), Variational Autoencoders (VAEs), and Diffusion Models. These models have each played a significant role in the evolution of generative modeling and continue to influence modern approaches in the field. 

\section{Noise Contrastive Estimation (NCEs)}

Noise Contrastive Estimation (NCE) was introduced in 2010 by Gutmann and Hyvärinen as a method for estimating parameters in unnormalized probabilistic models. It offers an efficient alternative to Maximum Likelihood Estimation (MLE), particularly in cases where MLE can become computationally expensive, especially with large-scale models. NCE reframes the challenge of normalizing probability distributions into a more manageable binary classification problem \citep{10.48550/arxiv.1711.00658}. 

The core idea behind NCE is to treat MLE as a binary classification task. Traditionally, training models for unnormalized probability distributions using MLE involves calculating the partition function, which can be computationally infeasible for large datasets. NCE mitigates this difficulty by incorporating noise samples drawn from a known distribution. The model is then trained to differentiate between real data and noise samples, with higher probabilities assigned to real data and lower probabilities to noise. This approach allows the model to learn the data distribution without the need for explicit normalization \citep{10.48550/arxiv.2110.11271}.

\subsection{NCE Architecture}
In NCE, the likelihood of a data point \( x \) is reformulated as the probability that it comes from the real data distribution rather than from the noise distribution. As depicted in Figure~\ref{fig:NCE_structure}, the architecture of a typical NCE model includes an input layer, a hidden layer, and an output layer. The input layer receives both real and noise samples, while the hidden layer learns representations of these samples. The output layer functions as a binary classifier, producing probabilities that indicate whether a given sample is real or noise.

\begin{figure}[H]
    \centering
    \includegraphics[width=0.9\textwidth]{./Images/NCE_structure.jpg}
    \caption{NCE Architecture.}
    \label{fig:NCE_structure}
\end{figure}

\begin{itemize}
    \item \textbf{Data and Noise Samples}: NCE introduces noise samples from a known distribution to compare with actual data. These noise samples act as negative examples in the classification task, while the real data serves as positive examples \citep{10.48550/arxiv.1711.00658}.
    \item \textbf{Binary Classification}: The task of differentiating between real and noise samples is framed as a binary classification problem, which can be optimized using standard logistic regression techniques \citep{10.18653/v1/e17-2003}.
\end{itemize}

\subsection{NCE Objective Function}
In NCE, the probability \( P(y = 1 | x) \), where \( y = 1 \) indicates that \( x \) is a real sample, is defined as:

\begin{equation}
P(y = 1 | x) = \frac{p_{\theta}(x)}{p_{\theta}(x) + k p_{\text{noise}}(x)}
\end{equation}

Where \( p_{\theta}(x) \) is the unnormalized probability assigned to the sample \( x \) by the model, \( p_{\text{noise}}(x) \) is the probability assigned to the noise sample, and \( k \) is the ratio of noise samples to real samples.

The corresponding probability that a sample \( x \) is from the noise distribution is given by:

\begin{equation}
P(y = 0 | x) = \frac{k p_{\text{noise}}(x)}{p_{\theta}(x) + k p_{\text{noise}}(x)}
\end{equation}

The NCE objective then seeks to maximize the log-probabilities of correctly classifying real and noise samples:

\begin{equation}
\mathcal{L}_{NCE} = \sum_{i=1}^{N} \left[ \log P(y=1 | x_i) + \sum_{j=1}^{k} \log P(y=0 | x_j) \right]
\end{equation}

This method circumvents the need for computing the partition function, which is typically required in Maximum Likelihood Estimation (MLE), making NCE particularly useful for large-scale models \citep{10.48550/arxiv.2110.11271}.
\subsection{Comparison with GANs}

Noise Contrastive Estimation (NCE) and Generative Adversarial Networks (GANs) are both prominent methods in generative modeling, but they take different approaches to training and model estimation.

\begin{itemize}
    \item \textbf{Training Stability}: One of NCE’s key advantages over GANs is the stability it provides during training. GANs frequently suffer from challenges such as mode collapse and instability due to the adversarial nature of the training process, where the generator and discriminator are pitted against each other. NCE, by contrast, frames the training process as a straightforward binary classification task, which tends to be more stable and can be optimized with logistic regression \citep{10.18653/v1/e17-2003}.
    
    \item \textbf{Computational Efficiency}: Training GANs requires simultaneously optimizing two models—the generator and the discriminator—which can lead to higher computational costs, particularly when tuning their interactions for better performance. NCE, on the other hand, simplifies the problem by reducing it to a comparison between real data and noise samples, avoiding the adversarial framework and offering a more computationally efficient solution, especially for large-scale models \citep{10.21437/interspeech.2016-1295}.
    
    \item \textbf{Handling Unnormalized Models}: NCE is particularly suited for training unnormalized probabilistic models, where computing the partition function is impractical or impossible. GANs, however, are focused on generating realistic samples and, while they do not require explicit normalization, they do not address the problem of unnormalized models as directly as NCE does \citep{10.48550/arxiv.2101.03288}.
    
    \item \textbf{Model Interpretability}: In NCE, the model explicitly learns to estimate the probability of real data versus noise, providing insights into the data distribution. In comparison, GANs focus on generating realistic samples without explicitly modeling probabilities, which can make their internal latent space less interpretable than that of NCE-based models.
    
    \item \textbf{Applications in Language Models and Word Embeddings}: NCE is particularly beneficial in tasks such as word embeddings and large-scale language models, where normalizing the likelihood function over vast vocabularies is computationally prohibitive. GANs have been applied in text generation, but NCE’s efficiency in handling large-scale vocabulary models makes it more suitable for these problems \citep{10.48550/arxiv.2101.03288}.
\end{itemize}

\subsection{Applications of NCE}

NCE is highly effective in a variety of machine learning tasks, particularly those that involve large datasets and unnormalized models:
\begin{itemize}
    \item \textbf{Word Embeddings}: NCE is widely used in training word embeddings, where the size of the vocabulary makes normalization computationally infeasible.
    \item \textbf{Language Models}: Beyond word embeddings, NCE has been applied to training large-scale language models, where traditional likelihood-based methods may become computationally expensive.
    \item \textbf{Energy-Based Models}: NCE is also effective in training energy-based models, which typically require an intractable partition function for normalization.
\end{itemize}

NCE allows models to scale efficiently, making it an invaluable tool in areas ranging from natural language processing to computer vision.

\subsection{Limitations of NCE}

While NCE has proven to be a powerful estimation technique, it is not without limitations. A key challenge lies in selecting an appropriate noise distribution. Poor choices in this regard can lead to suboptimal parameter estimates and slower convergence during training \citep{10.48550/arxiv.2110.11271}.

\begin{itemize}
    \item \textbf{Noise Distribution Sensitivity}: The success of NCE relies heavily on choosing a noise distribution that is sufficiently different from the real data distribution. If the noise distribution is poorly chosen, the model may struggle to differentiate between real and noise samples \citep{10.48550/arxiv.2110.11271}.
    \item \textbf{Handling Complex Data Distributions}: NCE may encounter difficulties with highly complex data distributions, particularly in cases where defining an appropriate noise distribution is challenging \citep{10.48550/arxiv.2110.11271}.
\end{itemize}


\section{Variational Autoencoders (VAEs)}

Variational Autoencoders (VAEs), introduced in 2013, represent one of the foundational approaches to generative modeling. The primary goal of VAEs is to model the underlying distribution of data by learning a compressed latent representation, denoted as \(z\), from the input data \(x\).

\subsection{VAE Architecture}
VAEs consist of two main components: an encoder and a decoder. The encoder maps input data into a latent space, where the latent variables are generally assumed to follow a Gaussian distribution. This assumption simplifies the training process and enables techniques such as the reparameterization trick, which facilitates backpropagation through stochastic layers \citep{10.1561/2200000056}. The decoder then reconstructs the input data from the latent variables, ensuring that the essential characteristics of the data distribution are captured.

As illustrated in Figure~\ref{fig:VAE_structure}, the VAE architecture comprises:
\begin{itemize}
    \item \textbf{Encoder}: The encoder takes the input data \(x\) and compresses it into a latent representation \(z\). The latent variables are sampled from a Gaussian distribution, which aids in simplifying the optimization process.
    \item \textbf{Decoder}: The decoder reconstructs the original data \(x'\) from the latent representation \(z\), aiming to produce data that closely resembles the input.
\end{itemize}

\begin{figure}[H]
    \centering
    \includegraphics[width=0.9\textwidth]{./Images/VAE_structure.jpg}
    \caption{VAE Architecture.}
    \label{fig:VAE_structure}
\end{figure}

\subsection{VAE Objective Function}
The VAE loss function is designed to balance two objectives: accurate data reconstruction and regularization of the latent space. The total loss combines a reconstruction loss with a Kullback-Leibler (KL) divergence term, which helps the model learn a smooth, continuous representation of the latent space \citep{10.3390/jimaging4020036}.

The VAE aims to maximize the variational lower bound, encouraging both accurate reconstruction and a well-structured latent space. This, in turn, enables the model to generate new data by sampling from the latent space and reconstructing it using the decoder.

The loss function of a VAE is expressed as follows:

\begin{equation}
\mathcal{L} = \mathbb{E}_{q_\phi(z|x)}[\log p_\theta(x|z)] - D_{KL}(q_\phi(z|x) \| p(z))
\end{equation}

Where:
\begin{itemize}
    \item \(\mathcal{L}\): The total loss that the model seeks to minimize.
    \item \(\mathbb{E}_{q_\phi(z|x)}[\log p_\theta(x|z)]\): This term represents the expected log-likelihood of the reconstructed data \(x'\) given the latent variable \(z\). The distribution \(q_\phi(z|x)\) corresponds to the encoder's approximation of the posterior, while \(p_\theta(x|z)\) represents the decoder's likelihood of reconstructing the input data.
    \item \(D_{KL}(q_\phi(z|x) \| p(z))\): The KL divergence measures the difference between the encoder's learned latent distribution \(q_\phi(z|x)\) and the prior distribution \(p(z)\), which is typically Gaussian.
\end{itemize}

\subsection{Comparison with GANs}

When compared to Generative Adversarial Networks (GANs), VAEs exhibit several advantages. One of the most notable strengths of VAEs is their training stability. In contrast to GANs, which involve training both a generator and a discriminator—a process that can lead to issues such as mode collapse—VAEs have a single objective function. This objective combines reconstruction loss and KL divergence, making the training process more straightforward and less prone to instability \citep{10.1561/2200000056}.

Another key strength of VAEs is their ability to generate smooth transitions between data points. This capability is particularly valuable for applications that require meaningful interpolation in the latent space, such as image generation, anomaly detection, and data imputation \citep{10.1088/2632-2153/ab80b7}\citep{10.48550/arxiv.2002.10464}. GANs, on the other hand, do not explicitly model the latent space, which can limit their interpretability and flexibility in certain tasks.

However, GANs are often favored for generating sharper and more realistic images, particularly in high-resolution tasks. The adversarial loss in GANs encourages the generator to produce outputs that closely resemble real data, while VAEs, due to their reliance on Gaussian priors, tend to generate slightly blurrier images \citep{10.1109/access.2020.2977671}. Nonetheless, VAEs offer greater flexibility and scalability, as they can be trained using standard gradient descent methods and do not require the complex adversarial framework inherent to GANs.

\subsection{Applications of VAEs}

The versatility of VAEs extends to a wide range of applications, many of which benefit from the structured latent space that VAEs provide:
\begin{itemize}
    \item \textbf{Image Generation}: VAEs can generate new images by sampling from the latent space and decoding the latent vectors into realistic image representations.
    \item \textbf{Anomaly Detection}: VAEs are often used to identify outliers in data by examining reconstruction errors. Data points with high reconstruction loss may be flagged as anomalies.
    \item \textbf{Data Imputation}: VAEs are capable of filling in missing data by reconstructing the incomplete data from latent representations, making them useful for handling datasets with gaps or missing values.
\end{itemize}

\subsection{Limitations of VAEs}

While VAEs have proven effective in many applications, they also have certain limitations:
\begin{itemize}
    \item \textbf{Blurred Outputs}: Due to the Gaussian prior assumption in the latent space, VAEs often generate blurrier outputs compared to GANs, which may affect performance in tasks requiring high-resolution and sharp images \citep{10.1109/access.2020.2977671}.
    \item \textbf{Difficulty with Discrete Data}: VAEs struggle to effectively model discrete data, as backpropagation through continuous latent variables is not well-suited to handle discrete structures \citep{10.48550/arxiv.1909.13062}.
    \item \textbf{Limited Diversity}: VAEs may not capture the full complexity and diversity of the data distribution as effectively as GANs, particularly when generating high-resolution images \citep{10.48550/arxiv.2106.06500}.
\end{itemize}

Despite these challenges, VAEs remain a foundational tool in generative modeling and continue to be widely used in unsupervised learning tasks, especially for applications requiring low-dimensional latent representations.

\section{Diffusion Models}

Diffusion models, emerging in the early 2020s, represent a significant advancement in generative modeling. These models progressively add noise to data and then learn to reverse this process, effectively "denoising" it back to its original form. This iterative process distinguishes diffusion models from traditional approaches like GANs and VAEs \citep{10.48550/arxiv.2105.05233}.

\subsection{Diffusion Model Architecture}

The core idea behind diffusion models relies on a Markov chain where noise is added in the forward process, starting from a simple distribution (e.g., Gaussian), and reversed through a learned denoising mechanism \citep{10.48550/arxiv.2009.09761}, \citep{10.48550/arxiv.2206.05564}. This approach has proven effective in generating high-quality samples across various domains, such as image synthesis, audio generation, and medical imaging \citep{10.48550/arxiv.2201.11972}, \citep{10.48550/arxiv.2211.00611}. In several benchmarks, diffusion models have outperformed GANs \citep{10.48550/arxiv.2105.05233}, \citep{10.48550/arxiv.2201.00308}.

The process involves two key steps, as shown in Figure~\ref{fig:Diffusion_structure}:
\begin{itemize}
  \item \textbf{Forward Process}: Gradually adds Gaussian noise to data \(x_0\), creating noisy versions of the data \(x_1, x_2, \dots, x_T\). Each step increases the noise level, leading to a completely noisy version \(z\).
  \item \textbf{Reverse Process}: The model learns to reverse the noise addition process, starting from the fully noisy version \(z\), and progressively denoising it to recover data that resembles the original input \(x_0\).
\end{itemize}

\begin{figure}[H]
    \centering
    \includegraphics[width=0.9\textwidth]{./Images/Diffusion_structure.jpg}
    \caption{Diffusion Model Architecture.}
    \label{fig:Diffusion_structure}
\end{figure}

\subsection{Diffusion Model Objective Function}

The training objective for diffusion models is to minimize the difference between the real data distribution and the distribution of generated data across all time steps:

\begin{equation}
L = \sum_{t=1}^{T} \mathbb{E}_{x_0, \epsilon} \left[ \|\epsilon - \epsilon_\theta(x_t, t)\|^2 \right]
\end{equation}

Where:
\begin{itemize}
    \item \(L\): The loss function to be minimized.
    \item \(T\): The total number of time steps in the diffusion process.
    \item \(x_0\): The original data sample.
    \item \(x_t\): The data at time step \(t\), after adding noise.
    \item \(\epsilon\): The noise added to the data at each step.
    \item \(\epsilon_\theta(x_t, t)\): The model's estimate of the noise at time step \(t\).
\end{itemize}

\subsection{Comparison with GANs}

Diffusion models and Generative Adversarial Networks (GANs) represent two distinct approaches in generative modeling, each with its strengths and weaknesses. A key advantage of diffusion models is their training stability. Unlike GANs, which often suffer from unstable dynamics due to the adversarial competition between the generator and discriminator, diffusion models follow a simpler, noise-reversal process that avoids these issues \citep{10.1109/access.2023.3272032}. This stability helps diffusion models avoid problems like mode collapse, where GANs fail to capture the diversity of the data distribution \citep{10.1049/ipr2.12487}, \citep{10.3390/e25121657}.

Diffusion models also excel in generating diverse, high-quality samples through a gradual denoising process, allowing for controlled generation \citep{10.1117/1.jei.32.4.043029}. In contrast, GANs tend to produce sharper but less diverse images, often overfitting to specific modes of the data \citep{10.48550/arxiv.1910.04302}, \citep{10.48550/arxiv.2207.01561}. This trade-off between sharpness and diversity is a known limitation of GANs \citep{10.1111/rssb.12476}.

However, GANs remain preferred for tasks requiring extremely high-resolution and photorealistic images, such as human face generation, where they outperform diffusion models in terms of detail and realism \citep{10.54254/2755-2721/18/20230984}, \citep{10.3390/e22091055}.

\subsection{Applications of Diffusion Models}

Diffusion models have found a wide range of applications, especially in fields where stability and quality of generation are important. Some notable applications include:
\begin{itemize}
    \item \textbf{Image Generation}: Diffusion models have proven effective in generating photorealistic images, similar to GANs, but with more stable training dynamics. Recent innovations, such as classifier-free guidance, have further enhanced the quality of generated samples by allowing the model to generate data without relying on an explicit classifier \citep{10.48550/arxiv.2207.12598}.
    \item \textbf{Text-to-Image Generation}: Advancements in diffusion models have also been applied to text-to-image synthesis, where a text prompt is converted into a corresponding image. These models can produce diverse outputs based on input descriptions. Additionally, techniques like Denoising Diffusion Implicit Models (DDIMs) have accelerated the sampling process, making diffusion models more practical for real-time applications \citep{10.48550/arxiv.2010.02502}, \citep{10.48550/arxiv.2111.15640}.
    \item \textbf{Speech Synthesis}: Diffusion models have been applied to generating high-quality audio data. For example, they are used in text-to-speech systems to generate realistic human speech. These models have demonstrated remarkable performance in generating high-fidelity audio outputs \citep{10.48550/arxiv.2201.11972}, \citep{10.48550/arxiv.2009.09761}.
    \item \textbf{Anomaly Detection and Medical Imaging}: Similar to VAEs, diffusion models can be used to detect anomalies by evaluating how well a noisy sample can be denoised. Poor reconstructions may indicate that the input is anomalous or different from the training data. Diffusion models have also been applied in medical image segmentation tasks, such as the MedSegDiff model, which enhances segmentation by leveraging diffusion processes \citep{10.48550/arxiv.2211.00611}. These models have shown their ability to handle complex data structures, proving their versatility across different domains.
\end{itemize}

Recent innovations in diffusion models have further enhanced their applicability and efficiency. By reducing the computational burden associated with the iterative sampling process, models like DDIMs maintain the generative capabilities of traditional diffusion models while improving their practicality for real-time applications \citep{10.48550/arxiv.2101.02388}.

\subsection{Limitations of Diffusion Models}

Despite the significant advancements that diffusion models have brought to generative modeling, they are not without limitations, which can be categorized into three primary areas: computational intensity, generation speed, and sample sharpness.

\begin{itemize}
    \item \textbf{Computationally Intensive}: Diffusion models are computationally expensive due to their iterative nature, which involves progressively adding and removing noise from data. This process requires significantly more computational resources compared to GANs, which generate data in a single forward pass through the generator \citep{10.1109/msp.2017.2765202}, \citep{10.1145/3422622}. This computational burden can limit their practical use, especially in scenarios requiring rapid generation \citep{10.48550/arxiv.2211.07804}.
    
    \item \textbf{Generation Speed}: Diffusion models are slower than GANs in terms of sample generation. Producing a single sample often requires hundreds or even thousands of time steps, significantly increasing the generation time compared to the near-instantaneous output of GANs \citep{10.48550/arxiv.2011.13456}, \citep{10.48550/arxiv.2010.02502}. This can be a critical drawback in real-time applications.
    
    \item \textbf{Sample Sharpness}: While diffusion models generate diverse outputs, they may not always match the photorealism and fine detail achieved by GANs, especially in tasks requiring intricate details \citep{10.48550/arxiv.2105.05233}, \citep{10.1109/cvpr52688.2022.01117}. This can affect their suitability for applications such as high-resolution image generation or medical imaging \citep{10.48550/arxiv.2211.07804}.
\end{itemize}


\newpage
\chapter{Theoretical Background}
\label{Theoretical Background}

\section*{Generative Adversarial Nets}

\newpage
\chapter{Experiments}
\label{Experiments}


\section*{Model select}
Approximately a decade has passed since Goodfellow introduced Generative Adversarial Networks (GANs), 
during which numerous variants of GAN models have been developed. For the purpose of this study, 
I have selected seven distinct GAN models for examination and minimal implementation: Standard GANs, 
Conditional GANs, Auxiliary Classifier GANs, Cycle GANs, Domain Transfer Network GANs, Coupled GANs, 
and Style GANs. The primary criterion for selection is training time, with only one model chosen for 
further in-depth analysis. Due to the extensive training time required, I ultimately selected Standard GANs for detailed study.


\section*{Model Structure}

In this step, I compared two types of standard GANs. The first model utilized dense layers, while the 
second model employed convolutional layers. The structure of the generator and discriminator was modified 
accordingly, while all other hyperparameters were kept constant. Both models were trained on the MNIST 
dataset for 1000 epochs. Upon comparing the generated images, it was observed that the GAN with the convolutional 
neural network (CNN) architecture outperformed the one with the dense layer architecture in terms of image quality. 
Consequently, I selected the standard GAN model implemented with convolutional layers for further experiments.

\begin{figure}[H]
    \centering
    \begin{subfigure}[b]{0.45\linewidth}
        \centering
        \includegraphics[width=\linewidth]{./Images/generator_dense.jpg}
        \caption{Generator with dense layer}
        \label{fig:Dense}
    \end{subfigure}
    \hspace{0.05\linewidth}
    \begin{subfigure}[b]{0.45\linewidth}
        \centering
        \includegraphics[width=\linewidth]{./Images/generator_cnn.jpg}
        \caption{Generator with convelution layer}
        \label{fig:Conv2D Transpose}
    \end{subfigure}
    \caption{The structure for generator}
    \label{fig:combined}
\end{figure}


\begin{figure}[H]
    \centering
    \begin{subfigure}[b]{0.45\linewidth}
        \centering
        \includegraphics[width=\linewidth]{./Images/discriminator_dense.jpg}
        \caption{Discriminator with dense layer}
        \label{fig:Dense}
    \end{subfigure}
    \hspace{0.05\linewidth}
    \begin{subfigure}[b]{0.45\linewidth}
        \centering
        \includegraphics[width=\linewidth]{./Images/discriminator_cnn.jpg}
        \caption{Discriminator with convelution layer}
        \label{fig:Conv2D Transpose}
    \end{subfigure}
    \caption{The structure for discriminator}
    \label{fig:combined}
\end{figure}


\begin{figure}[H]
    \centering
    \begin{subfigure}[b]{\linewidth}
        \centering
        \includegraphics[width=0.7\linewidth]{./Images/generate_image_by_dense_layer.jpg}
        \caption{Images generate by dense structure}
        \label{fig:Dense}
    \end{subfigure}
    \vspace{0.05\linewidth} 
    \begin{subfigure}[b]{\linewidth}
        \centering
        \includegraphics[width=0.7\linewidth]{./Images/generate_image_by_Convelution_layer.jpg}
        \caption{Images generate by convelution structure}
        \label{fig:Conv2DTranspose}
    \end{subfigure}
    \caption{Model performance 1000 epochs}
    \label{fig:combined}
\end{figure}

\section*{Data Augmentation}

I conducted three training runs of the standard GAN model, both with and without data augmentation, 
and compared the FID scores. The results indicate that data augmentation generally led to lower optimization performance.


\begin{figure}[H]
    \centering
    \begin{subfigure}[b]{\linewidth}
        \centering
        \includegraphics[width=0.8\linewidth]{./Images/standard_GAN_without_data_augementation1.jpg}
        \caption{Standard GAN without data augmentation 1}
        \label{fig:Dense}
    \end{subfigure}
    \vspace{0.05\linewidth} 
    \begin{subfigure}[b]{\linewidth}
        \centering
        \includegraphics[width=0.8\linewidth]{./Images/standard_GAN_without_data_augementation2.jpg}
        \caption{Standard GAN without data augmentation 2}
        \label{fig:Conv2DTranspose}
    \end{subfigure}
    \begin{subfigure}[b]{\linewidth}
        \centering
        \includegraphics[width=0.8\linewidth]{./Images/standard_GAN_without_data_augementation3.jpg}
        \caption{Standard GAN without data augmentation 3}
        \label{fig:Conv2DTranspose}
    \end{subfigure}
    \caption{The FID scores for standard GAN without data augmentation}
    \label{fig:combined}
\end{figure}


\begin{figure}[H]
    \centering
    \begin{subfigure}[b]{\linewidth}
        \centering
        \includegraphics[width=0.8\linewidth]{./Images/standard_GAN_with_data_augementation1.jpg}
        \caption{Standard GAN with data augmentation 1}
        \label{fig:Dense}
    \end{subfigure}
    \vspace{0.05\linewidth} 
    \begin{subfigure}[b]{\linewidth}
        \centering
        \includegraphics[width=0.8\linewidth]{./Images/standard_GAN_with_data_augementation2.jpg}
        \caption{Standard GAN with data augmentation 2}
        \label{fig:Conv2DTranspose}
    \end{subfigure}
    \begin{subfigure}[b]{\linewidth}
        \centering
        \includegraphics[width=0.8\linewidth]{./Images/standard_GAN_with_data_augementation3.jpg}
        \caption{Standard GAN with data augmentation 3}
        \label{fig:Conv2DTranspose}
    \end{subfigure}
    \caption{The FID scores for standard GAN with data augmentation}
    \label{fig:combined}
\end{figure}

\section*{Apply New Dataset}

\newpage
\section{Discussion}
\label{Discussion}

In this study, I conducted a series of experiments to evaluate the performance of different GAN architectures and explored various factors influencing the quality of generated images. Specifically, the experiments compared the effects of using dense layers versus convolutional layers in a standard GAN, examined the impact of varying the number of convolutional layers in both the generator and discriminator, analyzed the effects of data augmentation on GAN training, and applied a GAN model to a dataset to generate high-quality cat face images.

The experiments revealed that GAN models utilizing convolutional layers outperformed those using dense layers in terms of image quality, confirming the strength of convolutional layers in capturing spatial features. Additionally, I found that increasing the number of layers in the generator or discriminator independently led to a decline in performance, suggesting that the balance between these two components is critical for stable GAN training. Simultaneously increasing the number of layers in both components improved the model's output.

Regarding data augmentation, the results indicated that certain augmentation techniques negatively impacted GAN training, likely by introducing noise or disrupting the data distribution. This suggests that when applying data augmentation to a GAN, careful selection and parameter tuning are necessary to avoid adverse effects on training stability and performance.

When applying the model to the Animal Faces-HQ dataset, I successfully trained the GAN to generate cat images. However, due to hardware constraints, I had to downscale the images from 512x512 to 128x128 pixels, which may have limited the generated images' resolution and quality.

The limitations of this study include computational restrictions, which prevented experiments on higher-resolution images, and a focus on the standard GAN rather than more advanced architectures such as StyleGAN or BigGAN. Future work could explore these more complex GAN architectures and improve data augmentation strategies to enhance training performance and image quality. Additionally, training a GAN on higher-resolution datasets and experimenting with advanced GAN models, such as Conditional GAN or Adaptive GAN, could provide further insights into enhancing image diversity and fidelity.

In conclusion, this study demonstrates the effectiveness of GAN models in generating high-quality images, while highlighting areas for future improvements in architecture and training methods.

\newpage
\appendix
\chapter{Appendix}
\label{Code}
\section*{Code}

% First code block
\begin{lstlisting}[style=mypython, caption=GAN Model with Dense Layers]
import tensorflow as tf
from tensorflow.keras.layers import Dense, Reshape, Flatten, Dropout, LeakyReLU, BatchNormalization
from tensorflow.keras.models import Sequential
from tensorflow.keras.optimizers import Adam
from scipy.linalg import sqrtm
import numpy as np
import time
import matplotlib.pyplot as plt

np.random.seed(1000)
tf.random.set_seed(1000)

# input 100
# output 28*28*1

def build_generator():
    model = Sequential()
    
    # increase the dimension
    model.add(Dense(256, input_dim=100))
    model.add(LeakyReLU(alpha=0.2))
    model.add(BatchNormalization(momentum=0.8))
    
    model.add(Dense(512))
    model.add(LeakyReLU(alpha=0.2))
    model.add(BatchNormalization(momentum=0.8))
    
    model.add(Dense(1024))
    model.add(LeakyReLU(alpha=0.2))
    model.add(BatchNormalization(momentum=0.8))
    
    model.add(Dense(28*28, activation='tanh'))
    model.add(Reshape((28, 28, 1)))

    return model

# input 28*28*1
# output 1

def build_discriminator():
    model = Sequential()
    
    model.add(Flatten(input_shape=(28, 28, 1)))
    
    model.add(Dense(512))
    model.add(LeakyReLU(alpha=0.2))
    model.add(Dropout(0.25))
    
    model.add(Dense(256))
    model.add(LeakyReLU(alpha=0.2))
    model.add(Dropout(0.25))
    
    model.add(Dense(1, activation='sigmoid'))

    return model

# Load and preprocess the MNIST dataset
(x_train, _), (_, _) = tf.keras.datasets.mnist.load_data()
x_train = (x_train - 127.5) / 127.5
x_train = np.expand_dims(x_train, axis=3)

# Calculate FID function
def calculate_fid(real_images, fake_images):
    act1 = real_images.reshape((real_images.shape[0], -1))
    mu1, sigma1 = act1.mean(axis=0), np.cov(act1, rowvar=False)
    
    act2 = fake_images.reshape((fake_images.shape[0], -1))
    mu2, sigma2 = act2.mean(axis=0), np.cov(act2, rowvar=False)
    
    ssdiff = np.sum((mu1 - mu2)**2.0)
    covmean = sqrtm(sigma1.dot(sigma2))
    
    if np.iscomplexobj(covmean):
        covmean = covmean.real
    
    fid = ssdiff + np.trace(sigma1 + sigma2 - 2.0 * covmean)
    
    return fid

def train_gan(epochs=1000, batch_size=64, p_epoch=100):
    generator = build_generator()
    discriminator = build_discriminator()

    discriminator.compile(loss='binary_crossentropy', optimizer=Adam(0.0002, 0.5), metrics=['accuracy'])
    discriminator.trainable = False

    gan_input = tf.keras.Input(shape=(100,))
    gan_output = discriminator(generator(gan_input))
    gan = tf.keras.Model(gan_input, gan_output)
    gan.compile(loss='binary_crossentropy', optimizer=Adam(0.0002, 0.5))

    half_batch = int(batch_size / 2)
    
    d_losses = []
    g_losses = []
    d_acc = []
    fid_scores = []  # List to store FID scores
    
    start_time = time.time()  # Record the start time

    for epoch in range(epochs):
        # Select a random half batch of real images
        idx = np.random.randint(0, x_train.shape[0], half_batch)
        real_images = x_train[idx]

        # Generate a half batch of new fake images
        noise = np.random.normal(0, 1, (half_batch, 100))
        fake_images = generator.predict(noise)

        # Train the discriminator
        real_labels = np.ones((half_batch, 1))
        fake_labels = np.zeros((half_batch, 1))

        d_loss_real = discriminator.train_on_batch(real_images, real_labels)
        d_loss_fake = discriminator.train_on_batch(fake_images, fake_labels)
        d_loss = 0.5 * np.add(d_loss_real, d_loss_fake)

        # Train the generator
        noise = np.random.normal(0, 1, (batch_size, 100))
        valid_y = np.ones((batch_size, 1))

        g_loss = gan.train_on_batch(noise, valid_y)

        # Record the losses
        d_losses.append(d_loss[0])
        g_losses.append(g_loss)
        d_acc.append(d_loss[1] * 100)
        
        # Calculate and print FID every p_epoch epochs
        if epoch % p_epoch == 0:
            noise = np.random.normal(0, 1, (1000, 100))
            fake_images = generator.predict(noise)
            fid = calculate_fid(x_train[:1000], fake_images)
            fid_scores.append(fid)
            print(f"{epoch} [D loss: {d_loss[0]}, acc.: {100 * d_loss[1]}%] [G loss: {g_loss}] [FID: {fid}]")

    end_time = time.time()  # Record the end time
    total_time = end_time - start_time
    print(f"Total training time: {total_time:.2f} seconds")
    return generator, d_losses, g_losses, d_acc, fid_scores
\end{lstlisting}




\begin{lstlisting}[style=mypython, caption=GAN Model with Convolutional Layers]
    import tensorflow as tf
    from tensorflow.keras.layers import Dense, Reshape, Flatten, Dropout, LeakyReLU, Conv2D, Conv2DTranspose, BatchNormalization
    from tensorflow.keras.models import Sequential
    from tensorflow.keras.optimizers import Adam
    from scipy.linalg import sqrtm
    import numpy as np
    import time
    import matplotlib.pyplot as plt
    
    np.random.seed(1000)
    tf.random.set_seed(1000)
    
    # input 100
    # output 28*28*1
    
    def build_generator():
        model = Sequential()
        
        # increase the dimension
        model.add(Dense(7*7*128, input_dim=100))
        model.add(LeakyReLU(alpha=0.2))
        model.add(Reshape((7, 7, 128)))
        model.add(BatchNormalization(momentum=0.8))
    
        model.add(Conv2DTranspose(128, kernel_size=4, strides=2, padding='same'))
        model.add(LeakyReLU(alpha=0.2))
        model.add(BatchNormalization(momentum=0.8))
    
        model.add(Conv2DTranspose(64, kernel_size=4, strides=2, padding='same'))
        model.add(LeakyReLU(alpha=0.2))
        model.add(BatchNormalization(momentum=0.8))
    
        model.add(Conv2D(1, kernel_size=7, activation='tanh', padding='same'))
    
        return model
    
    # input 28*28*1
    # output 1
    
    def build_discriminator():
        model = Sequential()
        
        model.add(Conv2D(64, kernel_size=3, strides=2, input_shape=(28, 28, 1), padding='same'))
        model.add(LeakyReLU(alpha=0.2))
        model.add(Dropout(0.25))
    
        model.add(Conv2D(128, kernel_size=3, strides=2, padding='same'))
        model.add(LeakyReLU(alpha=0.2))
        model.add(Dropout(0.25))
    
        model.add(Flatten())
        model.add(Dense(1, activation='sigmoid'))
    
        return model
    
    # Load and preprocess the MNIST dataset
    (x_train, _), (_, _) = tf.keras.datasets.mnist.load_data()
    x_train = (x_train - 127.5) / 127.5
    x_train = np.expand_dims(x_train, axis=3)
    
    # Calculate FID function
    def calculate_fid(real_images, fake_images):
        act1 = real_images.reshape((real_images.shape[0], -1))
        mu1, sigma1 = act1.mean(axis=0), np.cov(act1, rowvar=False)
        
        act2 = fake_images.reshape((fake_images.shape[0], -1))
        mu2, sigma2 = act2.mean(axis=0), np.cov(act2, rowvar=False)
        
        ssdiff = np.sum((mu1 - mu2)**2.0)
        covmean = sqrtm(sigma1.dot(sigma2))
        
        if np.iscomplexobj(covmean):
            covmean = covmean.real
        
        fid = ssdiff + np.trace(sigma1 + sigma2 - 2.0 * covmean)
        
        return fid
    
    def train_gan(epochs=1000, batch_size=64, p_epoch=100):
        generator = build_generator()
        discriminator = build_discriminator()
    
        discriminator.compile(loss='binary_crossentropy', optimizer=Adam(0.0002, 0.5), metrics=['accuracy'])
        discriminator.trainable = False
    
        gan_input = tf.keras.Input(shape=(100,))
        gan_output = discriminator(generator(gan_input))
        gan = tf.keras.Model(gan_input, gan_output)
        gan.compile(loss='binary_crossentropy', optimizer=Adam(0.0002, 0.5))
    
        # using half batches for the discriminator ensures balanced and efficient training, 
        # better memory management, and more stable training dynamics in GANs.
        half_batch = int(batch_size / 2)
        
        d_losses = []
        g_losses = []
        d_acc = []
        fid_scores = []  # List to store FID scores
        
        start_time = time.time()  # Record the start time
    
        for epoch in range(epochs):
            # Select a random half batch of real images
            idx = np.random.randint(0, x_train.shape[0], half_batch)
            real_images = x_train[idx]
    
            # Generate a half batch of new fake images
            noise = np.random.normal(0, 1, (half_batch, 100))
            fake_images = generator.predict(noise)
    
            # Train the discriminator
            real_labels = np.ones((half_batch, 1))
            fake_labels = np.zeros((half_batch, 1))
    
            d_loss_real = discriminator.train_on_batch(real_images, real_labels)
            d_loss_fake = discriminator.train_on_batch(fake_images, fake_labels)
            d_loss = 0.5 * np.add(d_loss_real, d_loss_fake)
    
            # Train the generator
            noise = np.random.normal(0, 1, (batch_size, 100))
            valid_y = np.ones((batch_size, 1))
    
            g_loss = gan.train_on_batch(noise, valid_y)
    
            # Record the losses
            d_losses.append(d_loss[0])
            g_losses.append(g_loss)
            d_acc.append(d_loss[1] * 100)
            
            # Calculate FID every p_epoch epochs
            if epoch % p_epoch == 0:
                noise = np.random.normal(0, 1, (1000, 100))
                fake_images = generator.predict(noise)
                fid = calculate_fid(x_train[:1000], fake_images)
                fid_scores.append(fid)
                print(f"{epoch} [D loss: {d_loss[0]}, acc.: {100 * d_loss[1]}%] [G loss: {g_loss}] [FID: {fid}]")
        
        end_time = time.time()  # Record the end time
        total_time = end_time - start_time
        print(f"Total training time: {total_time:.2f} seconds")
        return generator, d_losses, g_losses, d_acc, fid_scores
\end{lstlisting}


\begin{lstlisting}[style=mypython, caption= { Explore Data Augmentaion (rotation 10, width and height shift 0.1 \\ and  horizontal flip)}, captionpos=t]
    import tensorflow as tf
    from tensorflow.keras.layers import Dense, Reshape, Flatten, Dropout, LeakyReLU, Conv2D, Conv2DTranspose, BatchNormalization
    from tensorflow.keras.models import Sequential
    from tensorflow.keras.optimizers import Adam
    from scipy.linalg import sqrtm
    import numpy as np
    import time
    import matplotlib.pyplot as plt
    
    np.random.seed(1000)
    tf.random.set_seed(1000)
    
    # input 100
    # output 28*28*1
    
    def build_generator():
        model = Sequential()
        
        # increase the dimension
        model.add(Dense(7*7*128, input_dim=100))
        model.add(LeakyReLU(alpha=0.2))
        model.add(Reshape((7, 7, 128)))
        model.add(BatchNormalization(momentum=0.8))
    
        model.add(Conv2DTranspose(128, kernel_size=4, strides=2, padding='same'))
        model.add(LeakyReLU(alpha=0.2))
        model.add(BatchNormalization(momentum=0.8))
    
        model.add(Conv2DTranspose(64, kernel_size=4, strides=2, padding='same'))
        model.add(LeakyReLU(alpha=0.2))
        model.add(BatchNormalization(momentum=0.8))
    
        model.add(Conv2D(1, kernel_size=7, activation='tanh', padding='same'))
    
        return model
    
    # input 28*28*1
    # output 1
    
    def build_discriminator():
        model = Sequential()
        
        model.add(Conv2D(64, kernel_size=3, strides=2, input_shape=(28, 28, 1), padding='same'))
        model.add(LeakyReLU(alpha=0.2))
        model.add(Dropout(0.25))
    
        model.add(Conv2D(128, kernel_size=3, strides=2, padding='same'))
        model.add(LeakyReLU(alpha=0.2))
        model.add(Dropout(0.25))
    
        model.add(Flatten())
        model.add(Dense(1, activation='sigmoid'))
    
        return model
    
    (x_train, _), (_, _) = tf.keras.datasets.mnist.load_data()
    x_train = (x_train - 127.5) / 127.5
    x_train = np.expand_dims(x_train, axis=3)
    
    datagen = tf.keras.preprocessing.image.ImageDataGenerator(
        rotation_range=10,
        width_shift_range=0.1,
        height_shift_range=0.1,
        horizontal_flip=True
    )
    
    def calculate_fid(real_images, fake_images):
        act1 = real_images.reshape((real_images.shape[0], -1))
        mu1, sigma1 = act1.mean(axis=0), np.cov(act1, rowvar=False)
        
        act2 = fake_images.reshape((fake_images.shape[0], -1))
        mu2, sigma2 = act2.mean(axis=0), np.cov(act2, rowvar=False)
        
        ssdiff = np.sum((mu1 - mu2)**2.0)
        covmean = sqrtm(sigma1.dot(sigma2))
        
        if np.iscomplexobj(covmean):
            covmean = covmean.real
        
        fid = ssdiff + np.trace(sigma1 + sigma2 - 2.0 * covmean)
        
        return fid
    
    def train_gan(epochs=1000, batch_size=64, p_epoch=100):
        generator = build_generator()
        discriminator = build_discriminator()
    
        discriminator.compile(loss='binary_crossentropy', optimizer=Adam(0.0002, 0.5), metrics=['accuracy'])
        discriminator.trainable = False
    
        gan_input = tf.keras.Input(shape=(100,))
        gan_output = discriminator(generator(gan_input))
        gan = tf.keras.Model(gan_input, gan_output)
        gan.compile(loss='binary_crossentropy', optimizer=Adam(0.0002, 0.5))
    
        half_batch = int(batch_size / 2)
        
        d_losses = []
        g_losses = []
        d_acc = []
        fid_scores = []
        
        start_time = time.time()  # Record the start time
    
        for epoch in range(epochs):
            # Select a random half batch of real images
            idx = np.random.randint(0, x_train.shape[0], half_batch)
            real_images = x_train[idx]
    
            real_images_augmented = next(datagen.flow(real_images, batch_size=half_batch))
    
            # Generate a half batch of new fake images
            noise = np.random.normal(0, 1, (half_batch, 100))
            fake_images = generator.predict(noise)
    
            # Train the discriminator
            real_labels = np.ones((half_batch, 1))
            fake_labels = np.zeros((half_batch, 1))
    
            d_loss_real = discriminator.train_on_batch(real_images_augmented, real_labels)
            d_loss_fake = discriminator.train_on_batch(fake_images, fake_labels)
            d_loss = 0.5 * np.add(d_loss_real, d_loss_fake)
    
            # Train the generator
            noise = np.random.normal(0, 1, (batch_size, 100))
            valid_y = np.ones((batch_size, 1))
    
            g_loss = gan.train_on_batch(noise, valid_y)
    
            # Record the losses
            d_losses.append(d_loss[0])
            g_losses.append(g_loss)
            d_acc.append(d_loss[1] * 100)
            
            # Calculate FID every p_epoch epochs
            if epoch % p_epoch == 0:
                noise = np.random.normal(0, 1, (1000, 100))
                fake_images = generator.predict(noise)
                fid = calculate_fid(x_train[:1000], fake_images)
                fid_scores.append(fid)
                print(f"{epoch} [D loss: {d_loss[0]}, acc.: {100 * d_loss[1]}%] [G loss: {g_loss}] [FID: {fid}]")
    
        end_time = time.time()  # Record the end time
        total_time = end_time - start_time
        print(f"Total training time: {total_time:.2f} seconds")
        return generator, d_losses, g_losses, d_acc, fid_scores
    
    # Training the GAN with data augmentation and FID calculation
    generator, d_losses, g_losses, d_acc, fid_scores = train_gan(epochs=1000, batch_size=64, p_epoch=100)
\end{lstlisting}


\begin{lstlisting}[style=mypython, caption={Explore Data Augmentation (rotation {10}, width and \\ height shift {0.1})},captionpos=t]
    import tensorflow as tf
    from tensorflow.keras.layers import Dense, Reshape, Flatten, Dropout, LeakyReLU, Conv2D, Conv2DTranspose, BatchNormalization
    from tensorflow.keras.models import Sequential
    from tensorflow.keras.optimizers import Adam
    from scipy.linalg import sqrtm
    import numpy as np
    import time
    import matplotlib.pyplot as plt
    
    np.random.seed(1000)
    tf.random.set_seed(1000)
    
    # input 100
    # output 28*28*1
    
    def build_generator():
        model = Sequential()
        
        # increase the dimension
        model.add(Dense(7*7*128, input_dim=100))
        model.add(LeakyReLU(alpha=0.2))
        model.add(Reshape((7, 7, 128)))
        model.add(BatchNormalization(momentum=0.8))
    
        model.add(Conv2DTranspose(128, kernel_size=4, strides=2, padding='same'))
        model.add(LeakyReLU(alpha=0.2))
        model.add(BatchNormalization(momentum=0.8))
    
        model.add(Conv2DTranspose(64, kernel_size=4, strides=2, padding='same'))
        model.add(LeakyReLU(alpha=0.2))
        model.add(BatchNormalization(momentum=0.8))
    
        model.add(Conv2D(1, kernel_size=7, activation='tanh', padding='same'))
    
        return model
    
    # input 28*28*1
    # output 1
    
    def build_discriminator():
        model = Sequential()
        
        model.add(Conv2D(64, kernel_size=3, strides=2, input_shape=(28, 28, 1), padding='same'))
        model.add(LeakyReLU(alpha=0.2))
        model.add(Dropout(0.25))
    
        model.add(Conv2D(128, kernel_size=3, strides=2, padding='same'))
        model.add(LeakyReLU(alpha=0.2))
        model.add(Dropout(0.25))
    
        model.add(Flatten())
        model.add(Dense(1, activation='sigmoid'))
    
        return model
    
    (x_train, _), (_, _) = tf.keras.datasets.mnist.load_data()
    x_train = (x_train - 127.5) / 127.5
    x_train = np.expand_dims(x_train, axis=3)
    
    datagen = tf.keras.preprocessing.image.ImageDataGenerator(
        rotation_range=10,
        width_shift_range=0.1,
        height_shift_range=0.1,
    )
    
    def calculate_fid(real_images, fake_images):
        act1 = real_images.reshape((real_images.shape[0], -1))
        mu1, sigma1 = act1.mean(axis=0), np.cov(act1, rowvar=False)
        
        act2 = fake_images.reshape((fake_images.shape[0], -1))
        mu2, sigma2 = act2.mean(axis=0), np.cov(act2, rowvar=False)
        
        ssdiff = np.sum((mu1 - mu2)**2.0)
        covmean = sqrtm(sigma1.dot(sigma2))
        
        if np.iscomplexobj(covmean):
            covmean = covmean.real
        
        fid = ssdiff + np.trace(sigma1 + sigma2 - 2.0 * covmean)
        
        return fid
    
    def train_gan(epochs=1000, batch_size=64, p_epoch=100):
        generator = build_generator()
        discriminator = build_discriminator()
    
        discriminator.compile(loss='binary_crossentropy', optimizer=Adam(0.0002, 0.5), metrics=['accuracy'])
        discriminator.trainable = False
    
        gan_input = tf.keras.Input(shape=(100,))
        gan_output = discriminator(generator(gan_input))
        gan = tf.keras.Model(gan_input, gan_output)
        gan.compile(loss='binary_crossentropy', optimizer=Adam(0.0002, 0.5))
    
        half_batch = int(batch_size / 2)
        
        d_losses = []
        g_losses = []
        d_acc = []
        fid_scores = []
        
        start_time = time.time()  # Record the start time
    
        for epoch in range(epochs):
            # Select a random half batch of real images
            idx = np.random.randint(0, x_train.shape[0], half_batch)
            real_images = x_train[idx]
    
            real_images_augmented = next(datagen.flow(real_images, batch_size=half_batch))
    
            # Generate a half batch of new fake images
            noise = np.random.normal(0, 1, (half_batch, 100))
            fake_images = generator.predict(noise)
    
            # Train the discriminator
            real_labels = np.ones((half_batch, 1))
            fake_labels = np.zeros((half_batch, 1))
    
            d_loss_real = discriminator.train_on_batch(real_images_augmented, real_labels)
            d_loss_fake = discriminator.train_on_batch(fake_images, fake_labels)
            d_loss = 0.5 * np.add(d_loss_real, d_loss_fake)
    
            # Train the generator
            noise = np.random.normal(0, 1, (batch_size, 100))
            valid_y = np.ones((batch_size, 1))
    
            g_loss = gan.train_on_batch(noise, valid_y)
    
            # Record the losses
            d_losses.append(d_loss[0])
            g_losses.append(g_loss)
            d_acc.append(d_loss[1] * 100)
            
            # Calculate FID every p_epoch epochs
            if epoch % p_epoch == 0:
                noise = np.random.normal(0, 1, (1000, 100))
                fake_images = generator.predict(noise)
                fid = calculate_fid(x_train[:1000], fake_images)
                fid_scores.append(fid)
                print(f"{epoch} [D loss: {d_loss[0]}, acc.: {100 * d_loss[1]}%] [G loss: {g_loss}] [FID: {fid}]")
    
        end_time = time.time()  # Record the end time
        total_time = end_time - start_time
        print(f"Total training time: {total_time:.2f} seconds")
        return generator, d_losses, g_losses, d_acc, fid_scores
    
    # Training the GAN with data augmentation and FID calculation
    generator, d_losses, g_losses, d_acc, fid_scores = train_gan(epochs=1000, batch_size=64, p_epoch=100)
\end{lstlisting}

\begin{lstlisting}[style=mypython, caption=Explore Data Augmetation (width and height shift {0.1})]
    import tensorflow as tf
    from tensorflow.keras.layers import Dense, Reshape, Flatten, Dropout, LeakyReLU, Conv2D, Conv2DTranspose, BatchNormalization
    from tensorflow.keras.models import Sequential
    from tensorflow.keras.optimizers import Adam
    from scipy.linalg import sqrtm
    import numpy as np
    import time
    import matplotlib.pyplot as plt
    
    np.random.seed(1000)
    tf.random.set_seed(1000)
    
    # input 100
    # output 28*28*1
    
    def build_generator():
        model = Sequential()
        
        # increase the dimension
        model.add(Dense(7*7*128, input_dim=100))
        model.add(LeakyReLU(alpha=0.2))
        model.add(Reshape((7, 7, 128)))
        model.add(BatchNormalization(momentum=0.8))
    
        model.add(Conv2DTranspose(128, kernel_size=4, strides=2, padding='same'))
        model.add(LeakyReLU(alpha=0.2))
        model.add(BatchNormalization(momentum=0.8))
    
        model.add(Conv2DTranspose(64, kernel_size=4, strides=2, padding='same'))
        model.add(LeakyReLU(alpha=0.2))
        model.add(BatchNormalization(momentum=0.8))
    
        model.add(Conv2D(1, kernel_size=7, activation='tanh', padding='same'))
    
        return model
    
    # input 28*28*1
    # output 1
    
    def build_discriminator():
        model = Sequential()
        
        model.add(Conv2D(64, kernel_size=3, strides=2, input_shape=(28, 28, 1), padding='same'))
        model.add(LeakyReLU(alpha=0.2))
        model.add(Dropout(0.25))
    
        model.add(Conv2D(128, kernel_size=3, strides=2, padding='same'))
        model.add(LeakyReLU(alpha=0.2))
        model.add(Dropout(0.25))
    
        model.add(Flatten())
        model.add(Dense(1, activation='sigmoid'))
    
        return model
    
    (x_train, _), (_, _) = tf.keras.datasets.mnist.load_data()
    x_train = (x_train - 127.5) / 127.5
    x_train = np.expand_dims(x_train, axis=3)
    
    datagen = tf.keras.preprocessing.image.ImageDataGenerator(
        width_shift_range=0.1,
        height_shift_range=0.1,
    )
    
    def calculate_fid(real_images, fake_images):
        act1 = real_images.reshape((real_images.shape[0], -1))
        mu1, sigma1 = act1.mean(axis=0), np.cov(act1, rowvar=False)
        
        act2 = fake_images.reshape((fake_images.shape[0], -1))
        mu2, sigma2 = act2.mean(axis=0), np.cov(act2, rowvar=False)
        
        ssdiff = np.sum((mu1 - mu2)**2.0)
        covmean = sqrtm(sigma1.dot(sigma2))
        
        if np.iscomplexobj(covmean):
            covmean = covmean.real
        
        fid = ssdiff + np.trace(sigma1 + sigma2 - 2.0 * covmean)
        
        return fid
    
    def train_gan(epochs=1000, batch_size=64, p_epoch=100):
        generator = build_generator()
        discriminator = build_discriminator()
    
        discriminator.compile(loss='binary_crossentropy', optimizer=Adam(0.0002, 0.5), metrics=['accuracy'])
        discriminator.trainable = False
    
        gan_input = tf.keras.Input(shape=(100,))
        gan_output = discriminator(generator(gan_input))
        gan = tf.keras.Model(gan_input, gan_output)
        gan.compile(loss='binary_crossentropy', optimizer=Adam(0.0002, 0.5))
    
        half_batch = int(batch_size / 2)
        
        d_losses = []
        g_losses = []
        d_acc = []
        fid_scores = []
        
        start_time = time.time()  # Record the start time
    
        for epoch in range(epochs):
            # Select a random half batch of real images
            idx = np.random.randint(0, x_train.shape[0], half_batch)
            real_images = x_train[idx]
    
            real_images_augmented = next(datagen.flow(real_images, batch_size=half_batch))
    
            # Generate a half batch of new fake images
            noise = np.random.normal(0, 1, (half_batch, 100))
            fake_images = generator.predict(noise)
    
            # Train the discriminator
            real_labels = np.ones((half_batch, 1))
            fake_labels = np.zeros((half_batch, 1))
    
            d_loss_real = discriminator.train_on_batch(real_images_augmented, real_labels)
            d_loss_fake = discriminator.train_on_batch(fake_images, fake_labels)
            d_loss = 0.5 * np.add(d_loss_real, d_loss_fake)
    
            # Train the generator
            noise = np.random.normal(0, 1, (batch_size, 100))
            valid_y = np.ones((batch_size, 1))
    
            g_loss = gan.train_on_batch(noise, valid_y)
    
            # Record the losses
            d_losses.append(d_loss[0])
            g_losses.append(g_loss)
            d_acc.append(d_loss[1] * 100)
            
            # Calculate FID every p_epoch epochs
            if epoch % p_epoch == 0:
                noise = np.random.normal(0, 1, (1000, 100))
                fake_images = generator.predict(noise)
                fid = calculate_fid(x_train[:1000], fake_images)
                fid_scores.append(fid)
                print(f"{epoch} [D loss: {d_loss[0]}, acc.: {100 * d_loss[1]}%] [G loss: {g_loss}] [FID: {fid}]")
    
        end_time = time.time()  # Record the end time
        total_time = end_time - start_time
        print(f"Total training time: {total_time:.2f} seconds")
        return generator, d_losses, g_losses, d_acc, fid_scores
    
    # Training the GAN with data augmentation and FID calculation
    generator, d_losses, g_losses, d_acc, fid_scores = train_gan(epochs=1000, batch_size=64, p_epoch=100)
\end{lstlisting}

\begin{lstlisting}[style=mypython, caption=Explore Data Augmetation (width and height shift {0.05})]
    import tensorflow as tf
    from tensorflow.keras.layers import Dense, Reshape, Flatten, Dropout, LeakyReLU, Conv2D, Conv2DTranspose, BatchNormalization
    from tensorflow.keras.models import Sequential
    from tensorflow.keras.optimizers import Adam
    from scipy.linalg import sqrtm
    import numpy as np
    import time
    import matplotlib.pyplot as plt
    
    np.random.seed(1000)
    tf.random.set_seed(1000)
    
    # input 100
    # output 28*28*1
    
    def build_generator():
        model = Sequential()
        
        # increase the dimension
        model.add(Dense(7*7*128, input_dim=100))
        model.add(LeakyReLU(alpha=0.2))
        model.add(Reshape((7, 7, 128)))
        model.add(BatchNormalization(momentum=0.8))
    
        model.add(Conv2DTranspose(128, kernel_size=4, strides=2, padding='same'))
        model.add(LeakyReLU(alpha=0.2))
        model.add(BatchNormalization(momentum=0.8))
    
        model.add(Conv2DTranspose(64, kernel_size=4, strides=2, padding='same'))
        model.add(LeakyReLU(alpha=0.2))
        model.add(BatchNormalization(momentum=0.8))
    
        model.add(Conv2D(1, kernel_size=7, activation='tanh', padding='same'))
    
        return model
    
    # input 28*28*1
    # output 1
    
    def build_discriminator():
        model = Sequential()
        
        model.add(Conv2D(64, kernel_size=3, strides=2, input_shape=(28, 28, 1), padding='same'))
        model.add(LeakyReLU(alpha=0.2))
        model.add(Dropout(0.25))
    
        model.add(Conv2D(128, kernel_size=3, strides=2, padding='same'))
        model.add(LeakyReLU(alpha=0.2))
        model.add(Dropout(0.25))
    
        model.add(Flatten())
        model.add(Dense(1, activation='sigmoid'))
    
        return model
    
    (x_train, _), (_, _) = tf.keras.datasets.mnist.load_data()
    x_train = (x_train - 127.5) / 127.5
    x_train = np.expand_dims(x_train, axis=3)
    
    datagen = tf.keras.preprocessing.image.ImageDataGenerator(
        width_shift_range=0.05,
        height_shift_range=0.05,
    )
    
    def calculate_fid(real_images, fake_images):
        act1 = real_images.reshape((real_images.shape[0], -1))
        mu1, sigma1 = act1.mean(axis=0), np.cov(act1, rowvar=False)
        
        act2 = fake_images.reshape((fake_images.shape[0], -1))
        mu2, sigma2 = act2.mean(axis=0), np.cov(act2, rowvar=False)
        
        ssdiff = np.sum((mu1 - mu2)**2.0)
        covmean = sqrtm(sigma1.dot(sigma2))
        
        if np.iscomplexobj(covmean):
            covmean = covmean.real
        
        fid = ssdiff + np.trace(sigma1 + sigma2 - 2.0 * covmean)
        
        return fid
    
    def train_gan(epochs=1000, batch_size=64, p_epoch=100):
        generator = build_generator()
        discriminator = build_discriminator()
    
        discriminator.compile(loss='binary_crossentropy', optimizer=Adam(0.0002, 0.5), metrics=['accuracy'])
        discriminator.trainable = False
    
        gan_input = tf.keras.Input(shape=(100,))
        gan_output = discriminator(generator(gan_input))
        gan = tf.keras.Model(gan_input, gan_output)
        gan.compile(loss='binary_crossentropy', optimizer=Adam(0.0002, 0.5))
    
        half_batch = int(batch_size / 2)
        
        d_losses = []
        g_losses = []
        d_acc = []
        fid_scores = []
        
        start_time = time.time()  # Record the start time
    
        for epoch in range(epochs):
            # Select a random half batch of real images
            idx = np.random.randint(0, x_train.shape[0], half_batch)
            real_images = x_train[idx]
    
            real_images_augmented = next(datagen.flow(real_images, batch_size=half_batch))
    
            # Generate a half batch of new fake images
            noise = np.random.normal(0, 1, (half_batch, 100))
            fake_images = generator.predict(noise)
    
            # Train the discriminator
            real_labels = np.ones((half_batch, 1))
            fake_labels = np.zeros((half_batch, 1))
    
            d_loss_real = discriminator.train_on_batch(real_images_augmented, real_labels)
            d_loss_fake = discriminator.train_on_batch(fake_images, fake_labels)
            d_loss = 0.5 * np.add(d_loss_real, d_loss_fake)
    
            # Train the generator
            noise = np.random.normal(0, 1, (batch_size, 100))
            valid_y = np.ones((batch_size, 1))
    
            g_loss = gan.train_on_batch(noise, valid_y)
    
            # Record the losses
            d_losses.append(d_loss[0])
            g_losses.append(g_loss)
            d_acc.append(d_loss[1] * 100)
            
            # Calculate FID every p_epoch epochs
            if epoch % p_epoch == 0:
                noise = np.random.normal(0, 1, (1000, 100))
                fake_images = generator.predict(noise)
                fid = calculate_fid(x_train[:1000], fake_images)
                fid_scores.append(fid)
                print(f"{epoch} [D loss: {d_loss[0]}, acc.: {100 * d_loss[1]}%] [G loss: {g_loss}] [FID: {fid}]")
    
        end_time = time.time()  # Record the end time
        total_time = end_time - start_time
        print(f"Total training time: {total_time:.2f} seconds")
        return generator, d_losses, g_losses, d_acc, fid_scores
    
    # Training the GAN with data augmentation and FID calculation
    generator, d_losses, g_losses, d_acc, fid_scores = train_gan(epochs=1000, batch_size=64, p_epoch=100)
\end{lstlisting}

\begin{lstlisting}[style=mypython, caption= {Explore GAN with 4 Convolutional Layers in Generator \\ and 3 Convolutional Layers in Discriminator}]
    import tensorflow as tf
    from tensorflow.keras.layers import Dense, Reshape, Flatten, Dropout, LeakyReLU, Conv2D, Conv2DTranspose, BatchNormalization
    from tensorflow.keras.models import Sequential
    from tensorflow.keras.optimizers import Adam
    from scipy.linalg import sqrtm
    import numpy as np
    import time
    import matplotlib.pyplot as plt
    
    np.random.seed(1000)
    tf.random.set_seed(1000)
    
    # input 100
    # output 28*28*1
    
    def build_generator():
        model = Sequential()
        
        model.add(Dense(7*7*128, input_dim=100))
        model.add(LeakyReLU(alpha=0.2))
        model.add(Reshape((7, 7, 128)))
        model.add(BatchNormalization(momentum=0.8))
    
        # add 1 convolution layer
        model.add(Conv2D(128, kernel_size=3, strides=1, padding='same'))
        model.add(LeakyReLU(alpha=0.2))
        model.add(BatchNormalization(momentum=0.8))
    
        model.add(Conv2DTranspose(128, kernel_size=4, strides=2, padding='same'))
        model.add(LeakyReLU(alpha=0.2))
        model.add(BatchNormalization(momentum=0.8))
    
        model.add(Conv2DTranspose(64, kernel_size=4, strides=2, padding='same'))
        model.add(LeakyReLU(alpha=0.2))
        model.add(BatchNormalization(momentum=0.8))
    
        model.add(Conv2D(1, kernel_size=7, activation='tanh', padding='same'))
    
        return model
    
    # input 28*28*1
    # output 1
    
    def build_discriminator():
        model = Sequential()
        
        model.add(Conv2D(64, kernel_size=3, strides=2, input_shape=(28, 28, 1), padding='same'))
        model.add(LeakyReLU(alpha=0.2))
        model.add(Dropout(0.25))
    
        model.add(Conv2D(128, kernel_size=3, strides=2, padding='same'))
        model.add(LeakyReLU(alpha=0.2))
        model.add(Dropout(0.25))
    
        model.add(Flatten())
        model.add(Dense(1, activation='sigmoid'))
    
        return model
    
    (x_train, _), (_, _) = tf.keras.datasets.mnist.load_data()
    x_train = (x_train - 127.5) / 127.5
    x_train = np.expand_dims(x_train, axis=3)
    
    def calculate_fid(real_images, fake_images):
        act1 = real_images.reshape((real_images.shape[0], -1))
        mu1, sigma1 = act1.mean(axis=0), np.cov(act1, rowvar=False)
        
        act2 = fake_images.reshape((fake_images.shape[0], -1))
        mu2, sigma2 = act2.mean(axis=0), np.cov(act2, rowvar=False)
        
        ssdiff = np.sum((mu1 - mu2)**2.0)
        covmean = sqrtm(sigma1.dot(sigma2))
        
        if np.iscomplexobj(covmean):
            covmean = covmean.real
        
        fid = ssdiff + np.trace(sigma1 + sigma2 - 2.0 * covmean)
        
        return fid
    
    def train_gan(epochs=1000, batch_size=64, p_epoch=100):
        generator = build_generator()
        discriminator = build_discriminator()
    
        discriminator.compile(loss='binary_crossentropy', optimizer=Adam(0.0002, 0.5), metrics=['accuracy'])
        discriminator.trainable = False
    
        gan_input = tf.keras.Input(shape=(100,))
        gan_output = discriminator(generator(gan_input))
        gan = tf.keras.Model(gan_input, gan_output)
        gan.compile(loss='binary_crossentropy', optimizer=Adam(0.0002, 0.5))
    
        half_batch = int(batch_size / 2)
        
        d_losses = []
        g_losses = []
        d_acc = []
        fid_scores = []
        
        start_time = time.time()  # Record the start time
    
        for epoch in range(epochs):
            # Select a random half batch of real images
            idx = np.random.randint(0, x_train.shape[0], half_batch)
            real_images = x_train[idx]
    
            # Generate a half batch of new fake images
            noise = np.random.normal(0, 1, (half_batch, 100))
            fake_images = generator.predict(noise)
    
            # Train the discriminator
            real_labels = np.ones((half_batch, 1))
            fake_labels = np.zeros((half_batch, 1))
    
            d_loss_real = discriminator.train_on_batch(real_images, real_labels)
            d_loss_fake = discriminator.train_on_batch(fake_images, fake_labels)
            d_loss = 0.5 * np.add(d_loss_real, d_loss_fake)
    
            # Train the generator
            noise = np.random.normal(0, 1, (batch_size, 100))
            valid_y = np.ones((batch_size, 1))
    
            g_loss = gan.train_on_batch(noise, valid_y)
    
            # Record the losses
            d_losses.append(d_loss[0])
            g_losses.append(g_loss)
            d_acc.append(d_loss[1] * 100)
            
            # Calculate FID every p_epoch epochs
            if epoch % p_epoch == 0:
                noise = np.random.normal(0, 1, (1000, 100))
                fake_images = generator.predict(noise)
                fid = calculate_fid(x_train[:1000], fake_images)
                fid_scores.append(fid)
                print(f"{epoch} [D loss: {d_loss[0]}, acc.: {100 * d_loss[1]}%] [G loss: {g_loss}] [FID: {fid}]")
    
        end_time = time.time()  # Record the end time
        total_time = end_time - start_time
        print(f"Total training time: {total_time:.2f} seconds")
        return generator, d_losses, g_losses, d_acc, fid_scores
\end{lstlisting}

\begin{lstlisting}[style=mypython, caption= {Explore GAN with 5 Convolutional Layers in Generator \\ and 3 Convolutional Layers in Discriminator}]
    import tensorflow as tf
    from tensorflow.keras.layers import Dense, Reshape, Flatten, Dropout, LeakyReLU, Conv2D, Conv2DTranspose, BatchNormalization
    from tensorflow.keras.models import Sequential
    from tensorflow.keras.optimizers import Adam
    from scipy.linalg import sqrtm
    import numpy as np
    import time
    import matplotlib.pyplot as plt
    
    np.random.seed(1000)
    tf.random.set_seed(1000)
    
    # input 100
    # output 28*28*1
    
    def build_generator():
        model = Sequential()
        
        model.add(Dense(7*7*128, input_dim=100))
        model.add(LeakyReLU(alpha=0.2))
        model.add(Reshape((7, 7, 128)))
        model.add(BatchNormalization(momentum=0.8))
    
        # add the first convolution layer
        model.add(Conv2D(128, kernel_size=3, strides=1, padding='same'))
        model.add(LeakyReLU(alpha=0.2))
        model.add(BatchNormalization(momentum=0.8))
    
        # add the 2nd convolution layer
        model.add(Conv2D(128, kernel_size=3, strides=1, padding='same'))
        model.add(LeakyReLU(alpha=0.2))
        model.add(BatchNormalization(momentum=0.8))
    
        model.add(Conv2DTranspose(128, kernel_size=4, strides=2, padding='same'))
        model.add(LeakyReLU(alpha=0.2))
        model.add(BatchNormalization(momentum=0.8))
    
        model.add(Conv2DTranspose(64, kernel_size=4, strides=2, padding='same'))
        model.add(LeakyReLU(alpha=0.2))
        model.add(BatchNormalization(momentum=0.8))
    
        model.add(Conv2D(1, kernel_size=7, activation='tanh', padding='same'))
    
        return model
    
    # input 28*28*1
    # output 1
    
    def build_discriminator():
        model = Sequential()
        
        model.add(Conv2D(64, kernel_size=3, strides=2, input_shape=(28, 28, 1), padding='same'))
        model.add(LeakyReLU(alpha=0.2))
        model.add(Dropout(0.25))
    
        model.add(Conv2D(128, kernel_size=3, strides=2, padding='same'))
        model.add(LeakyReLU(alpha=0.2))
        model.add(Dropout(0.25))
    
        model.add(Flatten())
        model.add(Dense(1, activation='sigmoid'))
    
        return model
    
    (x_train, _), (_, _) = tf.keras.datasets.mnist.load_data()
    x_train = (x_train - 127.5) / 127.5
    x_train = np.expand_dims(x_train, axis=3)
    
    def calculate_fid(real_images, fake_images):
        act1 = real_images.reshape((real_images.shape[0], -1))
        mu1, sigma1 = act1.mean(axis=0), np.cov(act1, rowvar=False)
        
        act2 = fake_images.reshape((fake_images.shape[0], -1))
        mu2, sigma2 = act2.mean(axis=0), np.cov(act2, rowvar=False)
        
        ssdiff = np.sum((mu1 - mu2)**2.0)
        covmean = sqrtm(sigma1.dot(sigma2))
        
        if np.iscomplexobj(covmean):
            covmean = covmean.real
        
        fid = ssdiff + np.trace(sigma1 + sigma2 - 2.0 * covmean)
        
        return fid
    
    def train_gan(epochs=1000, batch_size=64, p_epoch=100):
        generator = build_generator()
        discriminator = build_discriminator()
    
        discriminator.compile(loss='binary_crossentropy', optimizer=Adam(0.0002, 0.5), metrics=['accuracy'])
        discriminator.trainable = False
    
        gan_input = tf.keras.Input(shape=(100,))
        gan_output = discriminator(generator(gan_input))
        gan = tf.keras.Model(gan_input, gan_output)
        gan.compile(loss='binary_crossentropy', optimizer=Adam(0.0002, 0.5))
    
        half_batch = int(batch_size / 2)
        
        d_losses = []
        g_losses = []
        d_acc = []
        fid_scores = []
        
        start_time = time.time()  # Record the start time
    
        for epoch in range(epochs):
            # Select a random half batch of real images
            idx = np.random.randint(0, x_train.shape[0], half_batch)
            real_images = x_train[idx]
    
            # Generate a half batch of new fake images
            noise = np.random.normal(0, 1, (half_batch, 100))
            fake_images = generator.predict(noise)
    
            # Train the discriminator
            real_labels = np.ones((half_batch, 1))
            fake_labels = np.zeros((half_batch, 1))
    
            d_loss_real = discriminator.train_on_batch(real_images, real_labels)
            d_loss_fake = discriminator.train_on_batch(fake_images, fake_labels)
            d_loss = 0.5 * np.add(d_loss_real, d_loss_fake)
    
            # Train the generator
            noise = np.random.normal(0, 1, (batch_size, 100))
            valid_y = np.ones((batch_size, 1))
    
            g_loss = gan.train_on_batch(noise, valid_y)
    
            # Record the losses
            d_losses.append(d_loss[0])
            g_losses.append(g_loss)
            d_acc.append(d_loss[1] * 100)
            
            # Calculate FID every p_epoch epochs
            if epoch % p_epoch == 0:
                noise = np.random.normal(0, 1, (1000, 100))
                fake_images = generator.predict(noise)
                fid = calculate_fid(x_train[:1000], fake_images)
                fid_scores.append(fid)
                print(f"{epoch} [D loss: {d_loss[0]}, acc.: {100 * d_loss[1]}%] [G loss: {g_loss}] [FID: {fid}]")
    
        end_time = time.time()  # Record the end time
        total_time = end_time - start_time
        print(f"Total training time: {total_time:.2f} seconds")
        return generator, d_losses, g_losses, d_acc, fid_scores
    
\end{lstlisting}

\begin{lstlisting}[style=mypython, caption= {Explore GAN with 6 Convolutional Layers in Generator \\ and 3 Convolutional Layers in Discriminator}]
    import tensorflow as tf
    from tensorflow.keras.layers import Dense, Reshape, Flatten, Dropout, LeakyReLU, Conv2D, Conv2DTranspose, BatchNormalization
    from tensorflow.keras.models import Sequential
    from tensorflow.keras.optimizers import Adam
    from scipy.linalg import sqrtm
    import numpy as np
    import time
    import matplotlib.pyplot as plt
    
    np.random.seed(1000)
    tf.random.set_seed(1000)
    
    # input 100
    # output 28*28*1
    
    def build_generator():
        model = Sequential()
        
        model.add(Dense(7*7*128, input_dim=100))
        model.add(LeakyReLU(alpha=0.2))
        model.add(Reshape((7, 7, 128)))
        model.add(BatchNormalization(momentum=0.8))
    
        # add the 1st convolution layer
        model.add(Conv2D(128, kernel_size=3, strides=1, padding='same'))
        model.add(LeakyReLU(alpha=0.2))
        model.add(BatchNormalization(momentum=0.8))
    
        # add the 2nd convolution layer
        model.add(Conv2D(128, kernel_size=3, strides=1, padding='same'))
        model.add(LeakyReLU(alpha=0.2))
        model.add(BatchNormalization(momentum=0.8))
    
        # add the 3rd convolution layer
        model.add(Conv2D(128, kernel_size=3, strides=1, padding='same'))
        model.add(LeakyReLU(alpha=0.2))
        model.add(BatchNormalization(momentum=0.8))
    
        model.add(Conv2DTranspose(128, kernel_size=4, strides=2, padding='same'))
        model.add(LeakyReLU(alpha=0.2))
        model.add(BatchNormalization(momentum=0.8))
    
        model.add(Conv2DTranspose(64, kernel_size=4, strides=2, padding='same'))
        model.add(LeakyReLU(alpha=0.2))
        model.add(BatchNormalization(momentum=0.8))
    
        model.add(Conv2D(1, kernel_size=7, activation='tanh', padding='same'))
    
        return model
    
    # input 28*28*1
    # output 1
    
    def build_discriminator():
        model = Sequential()
        
        model.add(Conv2D(64, kernel_size=3, strides=2, input_shape=(28, 28, 1), padding='same'))
        model.add(LeakyReLU(alpha=0.2))
        model.add(Dropout(0.25))
    
        model.add(Conv2D(128, kernel_size=3, strides=2, padding='same'))
        model.add(LeakyReLU(alpha=0.2))
        model.add(Dropout(0.25))
    
        model.add(Flatten())
        model.add(Dense(1, activation='sigmoid'))
    
        return model
    
    (x_train, _), (_, _) = tf.keras.datasets.mnist.load_data()
    x_train = (x_train - 127.5) / 127.5
    x_train = np.expand_dims(x_train, axis=3)
    
    def calculate_fid(real_images, fake_images):
        act1 = real_images.reshape((real_images.shape[0], -1))
        mu1, sigma1 = act1.mean(axis=0), np.cov(act1, rowvar=False)
        
        act2 = fake_images.reshape((fake_images.shape[0], -1))
        mu2, sigma2 = act2.mean(axis=0), np.cov(act2, rowvar=False)
        
        ssdiff = np.sum((mu1 - mu2)**2.0)
        covmean = sqrtm(sigma1.dot(sigma2))
        
        if np.iscomplexobj(covmean):
            covmean = covmean.real
        
        fid = ssdiff + np.trace(sigma1 + sigma2 - 2.0 * covmean)
        
        return fid
    
    def train_gan(epochs=1000, batch_size=64, p_epoch=100):
        generator = build_generator()
        discriminator = build_discriminator()
    
        discriminator.compile(loss='binary_crossentropy', optimizer=Adam(0.0002, 0.5), metrics=['accuracy'])
        discriminator.trainable = False
    
        gan_input = tf.keras.Input(shape=(100,))
        gan_output = discriminator(generator(gan_input))
        gan = tf.keras.Model(gan_input, gan_output)
        gan.compile(loss='binary_crossentropy', optimizer=Adam(0.0002, 0.5))
    
        half_batch = int(batch_size / 2)
        
        d_losses = []
        g_losses = []
        d_acc = []
        fid_scores = []
        
        start_time = time.time()  # Record the start time
    
        for epoch in range(epochs):
            # Select a random half batch of real images
            idx = np.random.randint(0, x_train.shape[0], half_batch)
            real_images = x_train[idx]
    
            # Generate a half batch of new fake images
            noise = np.random.normal(0, 1, (half_batch, 100))
            fake_images = generator.predict(noise)
    
            # Train the discriminator
            real_labels = np.ones((half_batch, 1))
            fake_labels = np.zeros((half_batch, 1))
    
            d_loss_real = discriminator.train_on_batch(real_images, real_labels)
            d_loss_fake = discriminator.train_on_batch(fake_images, fake_labels)
            d_loss = 0.5 * np.add(d_loss_real, d_loss_fake)
    
            # Train the generator
            noise = np.random.normal(0, 1, (batch_size, 100))
            valid_y = np.ones((batch_size, 1))
    
            g_loss = gan.train_on_batch(noise, valid_y)
    
            # Record the losses
            d_losses.append(d_loss[0])
            g_losses.append(g_loss)
            d_acc.append(d_loss[1] * 100)
            
            # Calculate FID every p_epoch epochs
            if epoch % p_epoch == 0:
                noise = np.random.normal(0, 1, (1000, 100))
                fake_images = generator.predict(noise)
                fid = calculate_fid(x_train[:1000], fake_images)
                fid_scores.append(fid)
                print(f"{epoch} [D loss: {d_loss[0]}, acc.: {100 * d_loss[1]}%] [G loss: {g_loss}] [FID: {fid}]")
    
        end_time = time.time()  # Record the end time
        total_time = end_time - start_time
        print(f"Total training time: {total_time:.2f} seconds")
        return generator, d_losses, g_losses, d_acc, fid_scores
    
\end{lstlisting}

\begin{lstlisting}[style=mypython, caption= {Explore GAN with 6 Convolutional Layers in Generator \\ and 4 Convolutional Layers in Discriminator}]
    import tensorflow as tf
    from tensorflow.keras.layers import Dense, Reshape, Flatten, Dropout, LeakyReLU, Conv2D, Conv2DTranspose, BatchNormalization
    from tensorflow.keras.models import Sequential
    from tensorflow.keras.optimizers import Adam
    from scipy.linalg import sqrtm
    import numpy as np
    import time
    import matplotlib.pyplot as plt
    
    np.random.seed(1000)
    tf.random.set_seed(1000)
    
    # input 100
    # output 28*28*1
    
    def build_generator():
        model = Sequential()
        
        model.add(Dense(7*7*128, input_dim=100))
        model.add(LeakyReLU(alpha=0.2))
        model.add(Reshape((7, 7, 128)))
        model.add(BatchNormalization(momentum=0.8))
    
        # add the 1st convolution layer
        model.add(Conv2D(128, kernel_size=3, strides=1, padding='same'))
        model.add(LeakyReLU(alpha=0.2))
        model.add(BatchNormalization(momentum=0.8))
    
        # add the 2nd convolution layer
        model.add(Conv2D(128, kernel_size=3, strides=1, padding='same'))
        model.add(LeakyReLU(alpha=0.2))
        model.add(BatchNormalization(momentum=0.8))
    
        # add the 3rd convolution layer
        model.add(Conv2D(128, kernel_size=3, strides=1, padding='same'))
        model.add(LeakyReLU(alpha=0.2))
        model.add(BatchNormalization(momentum=0.8))
    
        model.add(Conv2DTranspose(128, kernel_size=4, strides=2, padding='same'))
        model.add(LeakyReLU(alpha=0.2))
        model.add(BatchNormalization(momentum=0.8))
    
        model.add(Conv2DTranspose(64, kernel_size=4, strides=2, padding='same'))
        model.add(LeakyReLU(alpha=0.2))
        model.add(BatchNormalization(momentum=0.8))
    
        model.add(Conv2D(1, kernel_size=7, activation='tanh', padding='same'))
    
        return model
    
    # input 28*28*1
    # output 1
    
    def build_discriminator():
        model = Sequential()
        
        model.add(Conv2D(64, kernel_size=3, strides=2, input_shape=(28, 28, 1), padding='same'))
        model.add(LeakyReLU(alpha=0.2))
        model.add(Dropout(0.25))
    
        model.add(Conv2D(128, kernel_size=3, strides=2, padding='same'))
        model.add(LeakyReLU(alpha=0.2))
        model.add(Dropout(0.25))
    
        # add the 1st convolution layer
        model.add(Conv2D(256, kernel_size=3, strides=2, padding='same'))
        model.add(LeakyReLU(alpha=0.2))
        model.add(Dropout(0.25))
    
        model.add(Flatten())
        model.add(Dense(1, activation='sigmoid'))
    
        return model
    
    (x_train, _), (_, _) = tf.keras.datasets.mnist.load_data()
    x_train = (x_train - 127.5) / 127.5
    x_train = np.expand_dims(x_train, axis=3)
    
    def calculate_fid(real_images, fake_images):
        act1 = real_images.reshape((real_images.shape[0], -1))
        mu1, sigma1 = act1.mean(axis=0), np.cov(act1, rowvar=False)
        
        act2 = fake_images.reshape((fake_images.shape[0], -1))
        mu2, sigma2 = act2.mean(axis=0), np.cov(act2, rowvar=False)
        
        ssdiff = np.sum((mu1 - mu2)**2.0)
        covmean = sqrtm(sigma1.dot(sigma2))
        
        if np.iscomplexobj(covmean):
            covmean = covmean.real
        
        fid = ssdiff + np.trace(sigma1 + sigma2 - 2.0 * covmean)
        
        return fid
    
    def train_gan(epochs=1000, batch_size=64, p_epoch=100):
        generator = build_generator()
        discriminator = build_discriminator()
    
        discriminator.compile(loss='binary_crossentropy', optimizer=Adam(0.0002, 0.5), metrics=['accuracy'])
        discriminator.trainable = False
    
        gan_input = tf.keras.Input(shape=(100,))
        gan_output = discriminator(generator(gan_input))
        gan = tf.keras.Model(gan_input, gan_output)
        gan.compile(loss='binary_crossentropy', optimizer=Adam(0.0002, 0.5))
    
        half_batch = int(batch_size / 2)
        
        d_losses = []
        g_losses = []
        d_acc = []
        fid_scores = []
        
        start_time = time.time()  # Record the start time
    
        for epoch in range(epochs):
            # Select a random half batch of real images
            idx = np.random.randint(0, x_train.shape[0], half_batch)
            real_images = x_train[idx]
    
            # Generate a half batch of new fake images
            noise = np.random.normal(0, 1, (half_batch, 100))
            fake_images = generator.predict(noise)
    
            # Train the discriminator
            real_labels = np.ones((half_batch, 1))
            fake_labels = np.zeros((half_batch, 1))
    
            d_loss_real = discriminator.train_on_batch(real_images, real_labels)
            d_loss_fake = discriminator.train_on_batch(fake_images, fake_labels)
            d_loss = 0.5 * np.add(d_loss_real, d_loss_fake)
    
            # Train the generator
            noise = np.random.normal(0, 1, (batch_size, 100))
            valid_y = np.ones((batch_size, 1))
    
            g_loss = gan.train_on_batch(noise, valid_y)
    
            # Record the losses
            d_losses.append(d_loss[0])
            g_losses.append(g_loss)
            d_acc.append(d_loss[1] * 100)
            
            # Calculate FID every p_epoch epochs
            if epoch % p_epoch == 0:
                noise = np.random.normal(0, 1, (1000, 100))
                fake_images = generator.predict(noise)
                fid = calculate_fid(x_train[:1000], fake_images)
                fid_scores.append(fid)
                print(f"{epoch} [D loss: {d_loss[0]}, acc.: {100 * d_loss[1]}%] [G loss: {g_loss}] [FID: {fid}]")
    
        end_time = time.time()  # Record the end time
        total_time = end_time - start_time
        print(f"Total training time: {total_time:.2f} seconds")
        return generator, d_losses, g_losses, d_acc, fid_scores
\end{lstlisting}

\begin{lstlisting}[style=mypython, caption= {Explore GAN with 6 Convolutional Layers in Generator \\ and 5 Convolutional Layers in Discriminator}]
    import tensorflow as tf
    from tensorflow.keras.layers import Dense, Reshape, Flatten, Dropout, LeakyReLU, Conv2D, Conv2DTranspose, BatchNormalization
    from tensorflow.keras.models import Sequential
    from tensorflow.keras.optimizers import Adam
    from scipy.linalg import sqrtm
    import numpy as np
    import time
    import matplotlib.pyplot as plt
    
    np.random.seed(1000)
    tf.random.set_seed(1000)
    
    # input 100
    # output 28*28*1
    
    def build_generator():
        model = Sequential()
        
        model.add(Dense(7*7*128, input_dim=100))
        model.add(LeakyReLU(alpha=0.2))
        model.add(Reshape((7, 7, 128)))
        model.add(BatchNormalization(momentum=0.8))
    
        # add the 1st convolution layer
        model.add(Conv2D(128, kernel_size=3, strides=1, padding='same'))
        model.add(LeakyReLU(alpha=0.2))
        model.add(BatchNormalization(momentum=0.8))
    
        # add the 2nd convolution layer
        model.add(Conv2D(128, kernel_size=3, strides=1, padding='same'))
        model.add(LeakyReLU(alpha=0.2))
        model.add(BatchNormalization(momentum=0.8))
    
        # add the 3rd convolution layer
        model.add(Conv2D(128, kernel_size=3, strides=1, padding='same'))
        model.add(LeakyReLU(alpha=0.2))
        model.add(BatchNormalization(momentum=0.8))
    
        model.add(Conv2DTranspose(128, kernel_size=4, strides=2, padding='same'))
        model.add(LeakyReLU(alpha=0.2))
        model.add(BatchNormalization(momentum=0.8))
    
        model.add(Conv2DTranspose(64, kernel_size=4, strides=2, padding='same'))
        model.add(LeakyReLU(alpha=0.2))
        model.add(BatchNormalization(momentum=0.8))
    
        model.add(Conv2D(1, kernel_size=7, activation='tanh', padding='same'))
    
        return model
    
    # input 28*28*1
    # output 1
    
    def build_discriminator():
        model = Sequential()
        
        model.add(Conv2D(64, kernel_size=3, strides=2, input_shape=(28, 28, 1), padding='same'))
        model.add(LeakyReLU(alpha=0.2))
        model.add(Dropout(0.25))
    
        model.add(Conv2D(128, kernel_size=3, strides=2, padding='same'))
        model.add(LeakyReLU(alpha=0.2))
        model.add(Dropout(0.25))
    
        model.add(Conv2D(256, kernel_size=3, strides=2, padding='same'))
        model.add(LeakyReLU(alpha=0.2))
        model.add(Dropout(0.25))
    
        # add the 1st convolution layer
        model.add(Conv2D(512, kernel_size=3, strides=2, padding='same'))
        model.add(LeakyReLU(alpha=0.2))
        model.add(Dropout(0.25))
    
        # add the 2nd convolution layer
        model.add(Conv2D(1024, kernel_size=3, strides=2, padding='same'))
        model.add(LeakyReLU(alpha=0.2))
        model.add(Dropout(0.25))
    
        model.add(Flatten())
        model.add(Dense(1, activation='sigmoid'))
    
        return model
    
    (x_train, _), (_, _) = tf.keras.datasets.mnist.load_data()
    x_train = (x_train - 127.5) / 127.5
    x_train = np.expand_dims(x_train, axis=3)
    
    def calculate_fid(real_images, fake_images):
        act1 = real_images.reshape((real_images.shape[0], -1))
        mu1, sigma1 = act1.mean(axis=0), np.cov(act1, rowvar=False)
        
        act2 = fake_images.reshape((fake_images.shape[0], -1))
        mu2, sigma2 = act2.mean(axis=0), np.cov(act2, rowvar=False)
        
        ssdiff = np.sum((mu1 - mu2)**2.0)
        covmean = sqrtm(sigma1.dot(sigma2))
        
        if np.iscomplexobj(covmean):
            covmean = covmean.real
        
        fid = ssdiff + np.trace(sigma1 + sigma2 - 2.0 * covmean)
        
        return fid
    
    def train_gan(epochs=1000, batch_size=64, p_epoch=100):
        generator = build_generator()
        discriminator = build_discriminator()
    
        discriminator.compile(loss='binary_crossentropy', optimizer=Adam(0.0002, 0.5), metrics=['accuracy'])
        discriminator.trainable = False
    
        gan_input = tf.keras.Input(shape=(100,))
        gan_output = discriminator(generator(gan_input))
        gan = tf.keras.Model(gan_input, gan_output)
        gan.compile(loss='binary_crossentropy', optimizer=Adam(0.0002, 0.5))
    
        half_batch = int(batch_size / 2)
        
        d_losses = []
        g_losses = []
        d_acc = []
        fid_scores = []
        
        start_time = time.time()  # Record the start time
    
        for epoch in range(epochs):
            # Select a random half batch of real images
            idx = np.random.randint(0, x_train.shape[0], half_batch)
            real_images = x_train[idx]
    
            # Generate a half batch of new fake images
            noise = np.random.normal(0, 1, (half_batch, 100))
            fake_images = generator.predict(noise)
    
            # Train the discriminator
            real_labels = np.ones((half_batch, 1))
            fake_labels = np.zeros((half_batch, 1))
    
            d_loss_real = discriminator.train_on_batch(real_images, real_labels)
            d_loss_fake = discriminator.train_on_batch(fake_images, fake_labels)
            d_loss = 0.5 * np.add(d_loss_real, d_loss_fake)
    
            # Train the generator
            noise = np.random.normal(0, 1, (batch_size, 100))
            valid_y = np.ones((batch_size, 1))
    
            g_loss = gan.train_on_batch(noise, valid_y)
    
            # Record the losses
            d_losses.append(d_loss[0])
            g_losses.append(g_loss)
            d_acc.append(d_loss[1] * 100)
            
            # Calculate FID every p_epoch epochs
            if epoch % p_epoch == 0:
                noise = np.random.normal(0, 1, (1000, 100))
                fake_images = generator.predict(noise)
                fid = calculate_fid(x_train[:1000], fake_images)
                fid_scores.append(fid)
                print(f"{epoch} [D loss: {d_loss[0]}, acc.: {100 * d_loss[1]}%] [G loss: {g_loss}] [FID: {fid}]")
    
        end_time = time.time()  # Record the end time
        total_time = end_time - start_time
        print(f"Total training time: {total_time:.2f} seconds")
        return generator, d_losses, g_losses, d_acc, fid_scores
\end{lstlisting}

\begin{lstlisting}[style=mypython, caption= {Explore GAN with 6 Convolutional Layers in Generator \\ and 6 Convolutional Layers in Discriminator}]
    def build_generator():
    model = Sequential()
    
    model.add(Dense(7*7*128, input_dim=100))
    model.add(LeakyReLU(alpha=0.2))
    model.add(Reshape((7, 7, 128)))
    model.add(BatchNormalization(momentum=0.8))

    # add the 1st convolution layer
    model.add(Conv2D(128, kernel_size=3, strides=1, padding='same'))
    model.add(LeakyReLU(alpha=0.2))
    model.add(BatchNormalization(momentum=0.8))

    # add the 2nd convolution layer
    model.add(Conv2D(128, kernel_size=3, strides=1, padding='same'))
    model.add(LeakyReLU(alpha=0.2))
    model.add(BatchNormalization(momentum=0.8))

    # add the 3rd convolution layer
    model.add(Conv2D(128, kernel_size=3, strides=1, padding='same'))
    model.add(LeakyReLU(alpha=0.2))
    model.add(BatchNormalization(momentum=0.8))

    model.add(Conv2DTranspose(128, kernel_size=4, strides=2, padding='same'))
    model.add(LeakyReLU(alpha=0.2))
    model.add(BatchNormalization(momentum=0.8))

    model.add(Conv2DTranspose(64, kernel_size=4, strides=2, padding='same'))
    model.add(LeakyReLU(alpha=0.2))
    model.add(BatchNormalization(momentum=0.8))

    model.add(Conv2D(1, kernel_size=7, activation='tanh', padding='same'))

    return model

def build_discriminator():
    model = Sequential()
    
    model.add(Conv2D(64, kernel_size=3, strides=2, input_shape=(28, 28, 1), padding='same'))
    model.add(LeakyReLU(alpha=0.2))
    model.add(Dropout(0.25))

    model.add(Conv2D(128, kernel_size=3, strides=2, padding='same'))
    model.add(LeakyReLU(alpha=0.2))
    model.add(Dropout(0.25))

    model.add(Conv2D(256, kernel_size=3, strides=2, padding='same'))
    model.add(LeakyReLU(alpha=0.2))
    model.add(Dropout(0.25))

   # add the 1st convolution layer
    model.add(Conv2D(512, kernel_size=3, strides=2, padding='same'))
    model.add(LeakyReLU(alpha=0.2))
    model.add(Dropout(0.25))

    # add the 2nd convolution layer
    model.add(Conv2D(1024, kernel_size=3, strides=2, padding='same'))
    model.add(LeakyReLU(alpha=0.2))
    model.add(Dropout(0.25))

    # add the 3rd convolution layer
    model.add(Conv2D(2048, kernel_size=3, strides=2, padding='same'))
    model.add(LeakyReLU(alpha=0.2))
    model.add(Dropout(0.25))

    model.add(Flatten())
    model.add(Dense(1, activation='sigmoid'))

    return model




import tensorflow as tf
from tensorflow.keras.layers import Dense, Reshape, Flatten, Dropout, LeakyReLU, Conv2D, Conv2DTranspose, BatchNormalization
from tensorflow.keras.models import Sequential
from tensorflow.keras.optimizers import Adam
from scipy.linalg import sqrtm
import numpy as np
import time
import matplotlib.pyplot as plt

np.random.seed(1000)
tf.random.set_seed(1000)

(x_train, _), (_, _) = tf.keras.datasets.mnist.load_data()
x_train = (x_train - 127.5) / 127.5
x_train = np.expand_dims(x_train, axis=3)

def calculate_fid(real_images, fake_images):
    act1 = real_images.reshape((real_images.shape[0], -1))
    mu1, sigma1 = act1.mean(axis=0), np.cov(act1, rowvar=False)
    
    act2 = fake_images.reshape((fake_images.shape[0], -1))
    mu2, sigma2 = act2.mean(axis=0), np.cov(act2, rowvar=False)
    
    ssdiff = np.sum((mu1 - mu2)**2.0)
    covmean = sqrtm(sigma1.dot(sigma2))
    
    if np.iscomplexobj(covmean):
        covmean = covmean.real
    
    fid = ssdiff + np.trace(sigma1 + sigma2 - 2.0 * covmean)
    
    return fid

def train_gan(epochs=1000, batch_size=64, p_epoch=100):
    generator = build_generator()
    discriminator = build_discriminator()

    discriminator.compile(loss='binary_crossentropy', optimizer=Adam(0.0002, 0.5), metrics=['accuracy'])
    discriminator.trainable = False

    gan_input = tf.keras.Input(shape=(100,))
    gan_output = discriminator(generator(gan_input))
    gan = tf.keras.Model(gan_input, gan_output)
    gan.compile(loss='binary_crossentropy', optimizer=Adam(0.0002, 0.5))

    half_batch = int(batch_size / 2)
    
    d_losses = []
    g_losses = []
    d_acc = []
    fid_scores = []
    
    start_time = time.time()  # Record the start time

    for epoch in range(epochs):
        # Select a random half batch of real images
        idx = np.random.randint(0, x_train.shape[0], half_batch)
        real_images = x_train[idx]

        # Generate a half batch of new fake images
        noise = np.random.normal(0, 1, (half_batch, 100))
        fake_images = generator.predict(noise)

        # Train the discriminator
        real_labels = np.ones((half_batch, 1))
        fake_labels = np.zeros((half_batch, 1))

        d_loss_real = discriminator.train_on_batch(real_images, real_labels)
        d_loss_fake = discriminator.train_on_batch(fake_images, fake_labels)
        d_loss = 0.5 * np.add(d_loss_real, d_loss_fake)

        # Train the generator
        noise = np.random.normal(0, 1, (batch_size, 100))
        valid_y = np.ones((batch_size, 1))

        g_loss = gan.train_on_batch(noise, valid_y)

        # Record the losses
        d_losses.append(d_loss[0])
        g_losses.append(g_loss)
        d_acc.append(d_loss[1] * 100)
        
        # Calculate FID every p_epoch epochs
        if epoch % p_epoch == 0:
            noise = np.random.normal(0, 1, (1000, 100))
            fake_images = generator.predict(noise)
            fid = calculate_fid(x_train[:1000], fake_images)
            fid_scores.append(fid)
            print(f"{epoch} [D loss: {d_loss[0]}, acc.: {100 * d_loss[1]}%] [G loss: {g_loss}] [FID: {fid}]")

    end_time = time.time()  # Record the end time
    total_time = end_time - start_time
    print(f"Total training time: {total_time:.2f} seconds")
    return generator, d_losses, g_losses, d_acc, fid_scores
\end{lstlisting}

\begin{lstlisting}[style=mypython, caption=Apply Animal Faces-HQ Dataset]

    import os
    import time
    import numpy as np
    import tensorflow as tf
    import matplotlib.pyplot as plt
    from tensorflow.keras.models import Sequential, Model
    from tensorflow.keras.layers import Dense, Reshape, BatchNormalization, LeakyReLU, Conv2D, Conv2DTranspose, Flatten, Dropout, Input
    from tensorflow.keras.optimizers import Adam
    from tensorflow.keras.preprocessing.image import load_img, img_to_array

    def load_images_as_rgb_matrices(path, size):
    images = []
    for img_name in os.listdir(path):
        img_path = os.path.join(path, img_name)
        img = load_img(img_path, target_size=size)
        img_array = img_to_array(img)
        images.append(img_array)
    images = np.array(images)
    images = (images - 127.5) / 127.5  # Normalize images to [-1, 1]
    return images

    from google.colab import drive

    # mount Google Drive
    drive.mount('/content/drive')

    cat_path = "/content/drive/My Drive/gan/afhq/train/cat"
    size = (128, 128)
    x_train = load_images_as_rgb_matrices(cat_path, size)

    # generator
    # input 100
    # output 128*128*3

    def build_generator():
        model = Sequential()

        # Increase the dimension
        model.add(Dense(16*16*256, input_dim=100))
        model.add(LeakyReLU(alpha=0.2))
        model.add(Reshape((16, 16, 256)))
        model.add(BatchNormalization(momentum=0.8))

        model.add(Conv2DTranspose(256, kernel_size=4, strides=2, padding='same'))  # 32x32
        model.add(LeakyReLU(alpha=0.2))
        model.add(BatchNormalization(momentum=0.8))

        model.add(Conv2DTranspose(128, kernel_size=4, strides=2, padding='same'))  # 64x64
        model.add(LeakyReLU(alpha=0.2))
        model.add(BatchNormalization(momentum=0.8))

        model.add(Conv2DTranspose(64, kernel_size=4, strides=2, padding='same'))  # 128x128
        model.add(LeakyReLU(alpha=0.2))
        model.add(BatchNormalization(momentum=0.8))

        model.add(Conv2D(3, kernel_size=7, activation='tanh', padding='same'))  # 128x128x3

    return model

    # discriminator
    # input 128*128*3
    # output 1

    def build_discriminator():
        model = Sequential()

        model.add(Conv2D(64, kernel_size=3, strides=2, input_shape=(128, 128, 3), padding='same'))
        model.add(LeakyReLU(alpha=0.2))
        model.add(Dropout(0.25))

        model.add(Conv2D(128, kernel_size=3, strides=2, padding='same'))
        model.add(LeakyReLU(alpha=0.2))
        model.add(Dropout(0.25))

        model.add(Conv2D(256, kernel_size=3, strides=2, padding='same'))
        model.add(LeakyReLU(alpha=0.2))
        model.add(Dropout(0.25))

        model.add(Conv2D(512, kernel_size=3, strides=2, padding='same'))
        model.add(LeakyReLU(alpha=0.2))
        model.add(Dropout(0.25))

        model.add(Flatten())
        model.add(Dense(1, activation='sigmoid'))

    return model

    # initial GAN
    def build_gan(generator, discriminator):
    discriminator.trainable = False
    gan_input = Input(shape=(100,))
    x = generator(gan_input)
    gan_output = discriminator(x)
    gan = Model(gan_input, gan_output)
    gan.compile(loss='binary_crossentropy', optimizer=tf.keras.optimizers.legacy.Adam(0.0002, 0.5))
    return gan

    # define training step
    def train(generator, discriminator, gan, x_train, epochs, batch_size=128):
        valid = np.ones((batch_size, 1))
        fake = np.zeros((batch_size, 1))

        for epoch in range(epochs):
            start_time = time.time()


            idx = np.random.randint(0, x_train.shape[0], batch_size)
            imgs = x_train[idx]

            noise = np.random.normal(0, 1, (batch_size, 100))
            gen_imgs = generator.predict(noise)

            d_loss_real = discriminator.train_on_batch(imgs, valid)
            d_loss_fake = discriminator.train_on_batch(gen_imgs, fake)
            d_loss = 0.5 * np.add(d_loss_real, d_loss_fake)


            noise = np.random.normal(0, 1, (batch_size, 100))
            g_loss = gan.train_on_batch(noise, valid)

            end_time = time.time()
            epoch_time = end_time - start_time

            print(f"{epoch} [D loss: {d_loss[0]} | D accuracy: {100*d_loss[1]}] [G loss: {g_loss}] [Epoch time: {epoch_time:.2f} seconds]")

    
    generator = build_generator()
    discriminator = build_discriminator()
    discriminator.compile(loss='binary_crossentropy', optimizer=tf.keras.optimizers.legacy.Adam(0.0002, 0.5), metrics=['accuracy'])
    gan = build_gan(generator, discriminator)

    train(generator, discriminator, gan, x_train, epochs=4000, batch_size=64)

    # gen images
    def show_generated_images(generator, num_images=25, dim=(5, 5), figsize=(10, 10)):
        noise = np.random.normal(0, 1, (num_images, 100))
        gen_imgs = generator.predict(noise)
        gen_imgs = 0.5 * gen_imgs + 0.5

        plt.figure(figsize=figsize)
        for i in range(num_images):
            plt.subplot(dim[0], dim[1], i+1)
            plt.imshow(gen_imgs[i])
            plt.axis('off')
        plt.tight_layout()
        plt.show()

    show_generated_images(generator)
\end{lstlisting}

\bibliographystyle{unsrt}
\bibliography{References}


\end{document}
