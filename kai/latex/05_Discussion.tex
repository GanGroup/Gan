\chapter{Discussion}
\label{Discussion}

\section{Summary of Key Findings}
In this study, I conducted a series of experiments to evaluate the performance of different 
GAN architectures and explored various factors influencing the quality of generated images. 
Specifically, I compared the effects of using dense layers versus convolutional layers in a standard GAN, 
examined the impact of varying the number of convolutional layers in the generator and discriminator, 
analyzed the effects of data augmentation on GAN training, and ultimately applied a GAN model to a 
new dataset to generate high-quality images of cat faces.

\section{Impact of Model Structure}
My experiments revealed that GAN models utilizing convolutional layers significantly outperformed
 those using dense layers in terms of image quality. This finding is consistent with the well-known 
 capability of convolutional layers to capture spatial features in images. Further experiments 
 demonstrated that increasing the number of layers in the generator or discriminator individually 
 led to decreased model performance. This imbalance disrupts the dynamic equilibrium between the 
 generator and discriminator in the GAN framework, underscoring the importance of maintaining this balance during GAN training.

\section{Data Augmentation Effects}
Regarding data augmentation, my results indicated that it generally led to lower optimization performance. 
Although data augmentation typically enhances a model's generalization ability, in this study, it might have 
introduced excessive noise, interfering with the GAN training process. This suggests that careful selection of 
augmentation methods and parameters is crucial to prevent destabilizing the model and hindering convergence during GAN training.

\section{Performance on New Dataset}
When applying the model to the high-resolution Animal Faces-HQ dataset, I successfully trained a standard GAN model 
to generate realistic cat images. This result validates the effectiveness of standard GANs in handling high-resolution 
images and demonstrates the potential of GANs in generating realistic images. However, due to hardware constraints, 
I had to downscale the images from 512x512 to 128x128 pixels, which might have limited the final quality of the generated images.

\section{Limitations and Future Work}
Despite the positive outcomes, this study has some limitations. Firstly, due to limited computational resources, 
I was unable to train the model on higher-resolution images, which may have affected the quality of the generated 
images. Secondly, my experiments primarily focused on standard GANs and their simple variants; future research 
could explore more advanced GAN variants such as StyleGAN or BigGAN. Additionally, in the area of data augmentation, 
further research is needed to design more effective augmentation strategies that can improve GAN training efficiency and the quality of generated images.

In future work, I plan to optimize the data augmentation strategies and attempt training on higher-resolution images. 
Furthermore, exploring different types of GAN models, such as Conditional GANs or Adaptive GANs, to enhance the diversity 
and quality of generated images could be valuable.

\section{Conclusion}
Overall, this study provides a comprehensive analysis of GAN performance through a series of experiments, demonstrating 
its potential in generating high-quality images. Despite some challenges, GANs remain a powerful tool in the field of 
generative models, and future improvements and optimizations are expected to further enhance their applicability.